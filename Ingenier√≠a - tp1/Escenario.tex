\documentclass[a4paper,10pt]{article}
\usepackage{graphicx}
\usepackage{verbatim}
\usepackage{subfig}
\usepackage{float}
\usepackage[spanish]{babel}   %ver bien como es
\usepackage[utf8]{inputenc}

\begin{document}
\section*{Escenarios}

El dia 29/02/2004 a las 12:47:59 entran simultaneamente al local de pizza Hack (sucursal barrio mitre) Alan Faina, 
Sergio Mazza, Tom\'as Anchorena y Laura Oliva en busqueda de una buena sapi. 
Alan se sienta en la mesa nro 1, Tomas en la mesa nro 2, Sergio Mazza en la 3 y Laura Oliva en la mesa 4. 

Alan se dispone a pedir su pizza preferida, la de anchoas, toma entonces el men\'u electr\'onico que se encuentra sobre su mesa y
mediante la secuencia de botones Comidas $\rightarrow$ Pizzas $\rightarrow$ Anchoas efectiviza su pedido y queda a la espera del mismo. El pedido se pudo 
realizar porque el monitereo de stock indicaba que la cantidad de anchoas era suficiente para realizar la pizza y, por lo tanto, en el men\'u
esta opci\'on le aparec\'ia a Alan como elegible. Sin embargo, el stock presente solo alcanzaba para una pizza de anchoas, por lo que (el agente
que moniterea el stock) le avisa al personal de la pizzeria que es hora de comprar m\'as anchoas.

Mientras tanto, en la mesa 2, Tomas Anchorena tambi\'en se dispone a pedir su pizza favorita que, al igual que Alan, se trata de la pizza de 
anchoas. Sin embargo, siendo Tomas tan pulcro, no postergo el lavado de sus manos por lo que cuando volvio a su mesa, Alan ya habia realizado
su pedido. Cuando entonces Tomas se dispone a pedir su ansiada pizza de anchoas, comienza a tocar la misma secuencia de botones que Alan. Sin embargo
cuando entra al men\'u de pizzas, se encuentra que el bot\'on correspondiente a la pizza de anchoas esta en gris (es decir, que no es seleccionable).
Tomas, desesperado por su pizza y conociendo el men\'u electr\'onico de la afamada pizzer\'ia, se fija si su pizza se encuentra disponible en otro
local. Pero, al igual que el bot\'on anterior, el bot\'on que indica la opci\'on para pedir pizza en otra sucursal, tampoco se encuentra disponible.

Tom\'as se retir\'a del local, triste por su pizza de anchoas, pero ya haciendosele agua la boca por la fugazzeta rellena de Banchero, su proxim\'o destino.




 
\begin{itemize}
\item pedir en otra sucursal e ir
\item pedir en otra y no ir
\item faltaria pedir stock
\end{itemize}
\end{document}
