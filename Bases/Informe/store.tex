Para realizar las consultas que fueron requeridas por las consignas del trabajo se decidió implementar las soluciones mediante Stored Procedures, para de esta forma poder acceder a las mismas de forma sencilla y cómoda.

\underline{Consulta:}

\textbf{El nombre de las salas que más declaraciones tomaron en el último año.}

Esta consulta es generada mediante el siguiente código:

\begin{verbatim}

SELECT nombre 'Nombre de la Sala', declaraciones 'Cantidad de Declaraciones'
FROM sala, (
	SELECT id_sala, count( * ) declaraciones
	FROM juzgado, (
		SELECT id_juzgado
		FROM secretaria, (
			SELECT id_secretaria
			FROM causas, (
				SELECT *
				FROM movimiento
				WHERE id
				IN (
					SELECT id_movimiento
					FROM declaraciones
				)
				) AS movConDec
			WHERE movConDec.id_causa = causas.id
			)secConDec
		WHERE secConDec.id_secretaria = secretaria.id
		) AS juzConDec
	WHERE juzConDec.id_juzgado = juzgado.id
	GROUP BY id_sala
	) AS salaConDec
WHERE sala.id = salaConDec.id_sala
ORDER BY declaraciones DESC
LIMIT 0 , 30

\end{verbatim}

La idea que se implemento es ir recorriendo desde las movimientos que fueron declaraciones hasta llegar a las Salas donde estos fueron originados, para ello se fue asociando las identificaciones correspondientes que enlazaban las diferentes tablas durante el camino. De esta forma se obtuvo una tabla con un campo sala y otro que indica la cantidad de declaraciones que fueron tomadas por la misma. Luego se procedió a ordenarlas de forma decreciente según la cantidad de declaraciones.

Se denomino \textbf{ranking()}, al Stored Procedure que implementa esta consulta.
\bigskip
\underline{Consulta:}

\textbf{El nombre de los jueces/secretarios que fueron nombrados por cada cámara a pesar de haber postulantes con un orden de mérito mejor que aún no habían sido designados.}

Esta consulta es generada mediante el siguiente código:

\begin{verbatim}
SELECT DISTINCT nombre
FROM    abogado AS a, concurso AS c, acargo, 
        inscripcion AS i, norma AS n, secretario AS s
WHERE   
    (a.cuil = acargo.cuil_abogado OR a.cuil = s.cuil_abogado) 
    -- El abogado a fue nombrado juez o secretario

    AND a.cuil = i.cuil_abogado 
    -- La inscripcion i corresponde al abogado a

    AND c.id = i.id_concurso  
    -- La inscripcion i corresponde al concurso c

    AND c.id IN 
    -- El concurso c es el que esta vigente para la norma n.
    -- Es decir, c es el ultimo concurso antes de que se publicara la norma n.
    (
    SELECT id
    FROM concurso AS conc
    WHERE   
        conc.fecha <= n.fecha_publicacion
        AND (conc.fecha >= ALL (
            SELECT fecha
            FROM concurso conc1
            WHERE conc1.fecha <= n.fecha_publicacion
            ))
    )
    AND EXISTS (    
    -- Existe un abogado inscripto en el mismo concurso,  
    -- con menor orden de merito, que
    -- no esta entre los nombrados jueces o secretarios.
    SELECT *
    FROM abogado AS abog, inscripcion AS insc
    WHERE   
        abog.cuil = insc.cuil_abogado 
        -- La inscripcion insc corresponde al abogado abog.
        AND insc.orden_merito < i.orden_merito  
        -- El orden de merito de la inscripcion insc
        -- es menor que el de la inscripcion i.
        AND insc.id_concurso = i.id_concurso    
        -- Las inscripciones insc e i 
        -- corresponden al mismo concurso
        AND NOT EXISTS (    
        -- No esta nombrado el abogado abog por una norma que 
        -- corresponda al concurso c.
            SELECT *
            FROM norma AS norm, acargo AS acargo2, secretario AS secretario2
            WHERE  
               norm.fecha_publicacion <= n.fecha_publicacion 
               -- La norma norm es anterior a la norma n
               AND norm.fecha_publicacion >= c.fecha    
               -- El concurso c ya estaba vigente cuando 
               -- se hizo la norma norm.
               AND 
               -- El abogado aparece nombrado juez o aparece nombrado 
               -- secretario bajo la norma norm.
                   (
                       (acargo2.cuil_abogado = abog.cuil 
                       -- El abogado abog aparece nombrado juez.
                       AND norm.id = acargo2.id_norma 
                       -- El nombramiento corresponde a la norma norm.
                       ) 
                    OR 
                       (secretario2.cuil_abogado = abog.cuil 
                       -- El abogado abog aparece nombrado secretario.
                       AND norm.id = secretario2.id_norma 
                       -- El nombramiento corresponde a la norma norm.
                       ) 
                   )
            )
    )
\end{verbatim}

Para resolver la consulta se identifican un abogado, la norma que lo nombra como juez o secretario y a su inscripción en el concurso vigente al momento de publicarse esa norma. Logrado esto, se buscan los abogados que tenían mejor posición en el orden de mérito del concurso vigente para los cuales no existía ninguna norma que los nombrara. Si este último conjunto no es vacío, entonces el abogado identificado fue nombrado habiendo todavía postulantes con un orden de mérito mejor que aún no habían sido designados.
Se denominó \textbf{acomodados()} al Stored Procedure con la intención de reflejar con un poco de humor la principal característica de los abogados cuyos nombres aparecen.

\bigskip
\underline{Control:}

\textbf{Registro automático de datos de control de las altas, bajas y modificaciones de concursos y sus resultados}

Para realizar esta tarea usamos la funcionalidad que nos provee el motor de la base de datos llamada "disparadores" o "triggers". La idea es llevar el control de los cambios que se realizan en las tablas relacionadas con la gestión de los concursos (Concurso e Inscripción). Para realizar esto creamos un conjunto de disparadores que registran los movimientos que se realizan en estas tablas y guardan dicha información en otras tablas creadas únicamente para este propósito.

El siguiente código implementa los triggers que mencionamos anteriormente:

\begin{verbatim}
CREATE TRIGGER auditoria_concurso_ins AFTER INSERT ON concurso
FOR EACH ROW
INSERT INTO auditoria_concurso(tipo, new_id_camara, new_id_concurso
,new_fecha, usuario, fecha)
VALUES ('Ins',NEW.id_camara, NEW.id,NEW.fecha, CURRENT_USER(), NOW());

CREATE TRIGGER auditoria_concurso_del AFTER DELETE ON concurso
FOR EACH ROW
INSERT INTO auditoria_concurso(tipo, old_id_camara, old_id_concurso,
old_fecha,usuario, fecha)
VALUES ('Del',OLD.id_camara, OLD.id,OLD.fecha, CURRENT_USER(), NOW());

CREATE TRIGGER auditoria_concurso_upd AFTER UPDATE ON concurso
FOR EACH ROW
INSERT INTO auditoria_concurso(tipo, new_id_camara, new_id_concurso,
old_id_camara,old_id_concurso, old_fecha,new_fecha,usuario, fecha)
VALUES ('Upd',NEW.id_camara, NEW.id,OLD.id_camara, OLD.id,OLD.fecha, 
NEW.fecha, CURRENT_USER(), NOW());

CREATE TRIGGER auditoria_inscripcion_ins AFTER INSERT ON inscripcion
FOR EACH ROW
INSERT INTO auditoria_inscripcion(tipo, new_id, new_orden_merito, 
new_nombre_universidad, new_promedio, new_fecha_titulo, new_id_concurso,
 new_cuil_abogado, usuario, fecha)
VALUES ('Ins',NEW.id, NEW.orden_merito, NEW.nombre_universidad,
 NEW.promedio, NEW.fecha_titulo, NEW.id_concurso, NEW.cuil_abogado,
  CURRENT_USER() , NOW());

CREATE TRIGGER auditoria_inscripcion_del AFTER DELETE ON inscripcion
FOR EACH ROW
INSERT INTO auditoria_inscripcion(tipo, old_id, old_orden_merito, 
old_nombre_universidad, old_promedio, old_fecha_titulo, old_id_concurso,
 old_cuil_abogado, usuario, fecha)
VALUES ('Del',OLD.id, OLD.orden_merito, OLD.nombre_universidad, 
OLD.promedio, OLD.fecha_titulo, OLD.id_concurso, OLD.cuil_abogado, 
CURRENT_USER() , NOW());

CREATE TRIGGER auditoria_inscripcion_upd AFTER UPDATE ON inscripcion
FOR EACH ROW
INSERT INTO auditoria_inscripcion(tipo, old_id, old_orden_merito,
old_nombre_universidad, old_promedio, old_fecha_titulo, old_id_concurso,
old_cuil_abogado,new_id, new_orden_merito, new_nombre_universidad, new_promedio,
new_fecha_titulo, new_id_concurso, new_cuil_abogado, usuario, fecha)
VALUES ('Upd',OLD.id, OLD.orden_merito, OLD.nombre_universidad, OLD.promedio,
OLD.fecha_titulo, OLD.id_concurso, OLD.cuil_abogado,NEW.id, NEW.orden_merito,
NEW.nombre_universidad, NEW.promedio, NEW.fecha_titulo, NEW.id_concurso,
NEW.cuil_abogado, CURRENT_USER() , NOW());

\end{verbatim}

