\documentclass[a4paper,11pt]{article}

\usepackage[spanish, activeacute]{babel}
\usepackage{caratula}
\usepackage{amsfonts}
\usepackage{color}
\usepackage{graphicx}
\usepackage{float}
\usepackage{ucs}
\usepackage[utf8x]{inputenc}
\usepackage{fontenc}
\usepackage{listings}
\usepackage{amssymb}
\usepackage{slashbox}
\usepackage{url} 
\usepackage[margin=2cm]{geometry}
\usepackage[bookmarks=true]{hyperref}


\usepackage{ulem}

\oddsidemargin  -0.0in
\evensidemargin -0.0in
\textwidth      6.3in
\headheight     0.0in
\topmargin      -0.4in
\textheight		9.5in

\begin{document}

\documentclass[a4paper,10pt]{article}
\usepackage{caratula}
\usepackage{a4wide}

%opening
\begin{document}
\titulo{Trabajo pr\'actico Nro. 1 Reentrega}
\fecha{29/06/2010}
\materia{Ingenier\'ia de Software I}
\integrante{Facundo Carrillo}{693/07}{facu.zeta@gmail.com}
\integrante{Rodrigo Casta\~{n}o}{602/07}{castano.rodrigo@gmail.com}
\integrante{Agustina Ciraco}{630/06}{agusciraco@gmail.com}
\integrante{Mart\'in De Micheli}{523/07}{shmdm7@gmail.com}
\integrante{Federico Pousa}{221/07}{fedepousa@gmail.com}

\maketitle

\end{document}




\tableofcontents

\newpage


\section{Heur\'isticas}

En esta secci\'on presentaremos las diferentes heur\'isticas que se aplicaron para el presente trabajo pr\'actico. 

Para implementar las heurísticas se tom\'o como punto de partida el hecho de que el input del programa sea un arbol can\'onico. Adem\'as, se destaca que para la implementaci\'on de cada heur\'istica se asume que el arbol ya sufri\'o los cambios de las heurísticas que la preceden.

\subsection{Cascada de Selecciones}

El input de esta heur\'istica es directamente el arbol can\'onico por lo que, al momento de ejecutarse la misma, se puede asumir que el arbol solo tendra un nodo de selecci\'on, donde es probable que la seleccion tenga condiciones multiples. La idea de esta heurística es entonces
poder partir este único nodo de selección con varias condiciones en muchos nodos de selección con una sola condicion en cada uno de los nodos. Esto es importante dado que luego, al estar cada selección por separado, las mismas se pueden manipular mejor para optimizar el arbol, ya sea bajando las selecciones lo
más cerca de las hojas posible, o bien juntandolas con los productos cartesianos para generar natural joins.

La implementación de esta heurística es bastante straightforward, simplemente se levanta la lista de condiciones del único nodo de selección y luego se va creando un nodo de selección por cada condicion de esta lista. Por último, se reemplaza el nodo original del arbol por esta nueva cascada de selecciones.


\subsection{Intercambiar hojas}

El input de esta heurística es un arbol optimizado, donde la única optimización realizada hasta el momento, fue la separación de las condiciones de selección en varios nodos diferentes. Asumir este paso es importante ya que si las condiciones se encontrasen todas juntas lo más probable es que esa selección no coincida con las condiciones necesarias para realizar un natural joins, dado que es probable que la selección original tenga condiciones sobre varias tablas y no solo sobre igualdades. La idea de esta heurística es entonces poder ver si se pueden cambiar las hojas entre sí para que luego, en la aplicación de las heurísticas siguientes, se puedan cambiar condiciones de seleccion y productos cartesianos por natural joins lo cual representa, en la mayoría de los casos, una gran mejora para el costo de procesamiento de la consulta.

Para implementar esta heurística se listaron todas las condiciones de junta para ver que tablas se encontraban afectadas por las mismas. Una vez que se encuentran dos tablas afectadas por una condicion de junta, se pegan juntas en el arbol con las primeras dos hojas desde más a la izquierda y luego se sigue viendo si existen otras condiciones de junta para esas mismas relaciones. Luego, se sigue procesando las demás tablas para ver si existen condiciones de junta para relacionarlas con las tablas que ya fueron utilizadas.

Para la implementación de esta heurística se utilizaron varios métodos auxiliares para listar las tablas, listar todas las condiciones, saber cuales de estas son condiciones de junta, entre otras funcionalidades requeridas. Todos estos se métodos se encuentran implementados en la clase TreeHelper.java

\subsection{Bajar selecciones con condiciones de junta}

El input de esta heurística es un arbol optimizado, donde el único nodo de selección del arbol canónico ya fue partido en varias selecciones diferentes y además las hojas ya fueron intercambiadas en base a las condiciones de junta presentes. Luego, lo que se quiere lograr en este paso es poder identificar dichas condiciones de junta para así bajarlas hasta el producto cartesiano que le corresponde para que luego, en la próxima heurística, dicha condición y dicho producto se puedan juntar y formar un join. 

Para implementar esta heurística se listaron todas las condiciones de junta presentes en el arbol, utilizando el método auxiliar creado para la heurística anterior. Una vez obtenidas dichas condiciones, se itera sobre el arbol hasta encontrar productos, a medida que se van encontrando los productos se va chequeando si los hijos de ese producto son los correspondientes a las condiciones de junta listadas anteriormente, de ser así se posiciona la condición de junta pertinente como nodo padre del producto que se estaba analizando, de lo contrario se sigue con la iteración.

\subsection{Reemplazar productos cartesianos}

El input de esta heurística es el arbol optimizado dado como output de la heurística anterior por lo que se puede asumir que todas las condiciones de junta necesarias para cambiar los productos cartesianos por joins se encuentran encima de los productos correspondientes. Dicho esto, lo único que se tiene que hacer en la heurística es cambiar estas condiciones de junta y los productos cartesianos por nodos nuevos correspondientes a los joins.

La implementación de la heurística es bastante directa dada la fuerte asunción sobre el input, lo que hace la misma es iterar sobre el arbol buscando los productos, una vez que encuentra un producto se fija si los nodos superiores al mismo corresponden a condiciones de junta de los hijos del producto que se esta analizando. En caso afirmativo se pueden reemplazar por un join. 

\subsection{Bajar Selecciones}

El input de esta heur\'istica es el arbol que se pudieron bajar las selecciones que se juntan con los productos para cambiarlos por juntas. Es por esto, que para esta heur\'istica se asume que todas las selecciones que existen todav\'ia en el arbol, se presentan como una secuencia de selecciones simples luego de la \'unica proyeccion y antes del primer producto o junta. 

La implementaci\'on de esta heur\'istica se basa en recortar todas las selecciones anteriormente mencionadas y se van bajando recursivamente hacia las diferentes ramas del arbol, chequeando previamente que selecciones se deben bajar por cuales ramas. Cuando se llega al primer producto o junta, se procesa la rama derecha que, dado que se tratan de arboles sesgados a izquierda, se sabe que es una relaci\'on. Por lo cual, lo \'unico que se tiene que hacer es ver cuales de las selecciones que se hab\'ian recolectado al bajar por arbol corresponden a dicha tabla. Luego, se sigue recursivamente con el algoritmo por la rama izquierda.




\subsection{Bajar Proyecciones}

Para esta \'ultima he\'uristica se tomaron las siguientes decisiones.

\begin{itemize}
 \item Crear proyecciones nuevas despues de las selecciones: Se crearon nuevos nodos de proyecci\'on arriba de los nodos de selecci\'on para que solamente perduren los campos necesarios en los nodos m\'as altos del arbol. Se podr\'ia haber puesto nuevas proyecciones en cada uno de los nodos del arbol de entrada siempre que se pudiese achicar algo, pero se tom\'o esta decisi\'on dado que de otra manera el arbol podr\'ia quedar muy sobrecargado, quedando un arbol de 2n nodos, con n proyecciones. Esta desici\'on es tan solo eso, ser\'ia bueno ver como una extensi\'on a este trabajo si la desici\'on tomada es la mejor.

\item Crear proyecciones antes de las relaciones, proyectando solamente los campos que son utilizados en alg\'un nodo del arbol. Es importante destacar que no se pueden desprender cuales son los campos de una relaci\'on por lo que siempre se proyectan todos los campos usados en el arbol para esa relaci\'on. La soluci\'on a esto ser\'ia trabajar con el cat\'alogo de la base, pero en este caso, solo agrega complejidad sin añadir ninguna mejora a la idea del tp.

\end{itemize}

Para la implementaci\'on de esta heur\'istica lo que se hiz\'o fue implementar una funci\'on que se aplica recursivamente desde la ra\'iz del arbol hasta las relaciones. La funci\'on identifica que tipo de nodo se esta trabajando y actua en consecuencia, si se trata de una selecci\'on o una relaci\'on se crea un nodo de proyecci\'on intermedio con los campos pertinentes para esa rama del arbol.




\section{Casos de Test}


\end{document}
