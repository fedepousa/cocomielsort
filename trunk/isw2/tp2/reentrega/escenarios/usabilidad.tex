\subsection{Usabilidad}

\begin{enumerate}

%Adrian: Que se pueda votar desde diversas plataformas, celulares, todo!. (extensibilidad, modificabilidad, funcionalidad, flexiblidad)
\item \escenario
{Que se pueda votar desde diversas plataformas, celulares, todo.}
{Un votante}
{Se intenta votar desde una Tablet conectada con 4G por internet.}
{Normal}
{Sistema web}
{El sistema permite votar siendo transparente la plataforma para \'este.}
{El sistema funciona correctamente en los 5 dispositivos m\'as utilizados.}


\item  \escenario{Se quiere que el sistema sea sencillo de usar, para que no se confunda al electorado al momento de emitir el voto}{Usuario}{El usuario ingresa el voto para el candidato que eligió}{Operación normal}{Sistema}{Se registra el voto del usuario}{El 100 \% de los usuarios efectivamente votó al usuario que cree haber votado.}

\medskip




\item  \escenario
{La interfaz debe ser simple, para que no consuma mucho tiempo del usuario}
{Usuario}
{El usuario ingresa el voto para el candidato que eligió}
{Operación normal}
{Sistema}
{Se registra el voto del usuario}
{El 99\% de los usarios logran emitir su voto en dos minutos.}
\medskip



\medskip
\item   \escenario{Usable en otros idiomas!, fácil de configurar el idioma}
{Un votante}
{Un votante que domina mal el espa\~nol ingresa el sitio para votar y desea cambiar el idioma a ingl\'es}
{Operación Normal}
{Sistema}
{El sistema permite elegir el idioma y cambiar sus contenidos al idioma correspondiente.}
{El usuario encuentra c\'omo cambiar el idioma en menos de 15 segundos.}
\medskip


\end{enumerate}
