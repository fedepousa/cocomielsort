\documentclass[a4paper,11pt]{article}

\usepackage{caratula}
\usepackage[spanish, activeacute]{babel}
\usepackage{amsfonts}
\usepackage{color}
\usepackage{graphicx}
\usepackage{float}
%\usepackage{algorithm2e}
% \usepackage{algpseudocode}
\usepackage{ucs}
\usepackage[utf8x]{inputenc}
\usepackage{fontenc}
\usepackage{listings}
\usepackage{amssymb}
\usepackage{amsmath}
%~ \usepackage{slashbox}
\usepackage{url} 
\usepackage[margin=2cm]{geometry}
\usepackage[bookmarks=true]{hyperref}

\renewcommand{\labelitemi}{}
\renewcommand{\labelitemii}{}
\renewcommand{\labelitemiii}{}
\renewcommand{\labelitemiv}{}



\begin{document}


\titulo{Trabajo pr\'actico Nro. 2}
\subtitulo{\emph{Programming in the large}}
\fecha{\today}
\materia{Ingenier\'ia de Software II}
\integrante{Facundo Carillo}{221/07}{fedepousa@gmail.com}
\integrante{Rodrigo Casta\~no}{221/07}{fedepousa@gmail.com}
\integrante{Brian Curcio}{61/07}{bcurcio@gmail.com}
\integrante{Federico Pousa}{221/07}{fedepousa@gmail.com}
\integrante{Felipe Schargorosdky}{221/07}{fedepousa@gmail.com}

\maketitle


 %1 Carátula + Abstract.

\newpage

\tableofcontents

\newpage 

%Hay que agregar casos de uso del TP1 muchos, tambien tiene muchas correciones 
\section{Casos de Uso}

A continuaci\'on se presentan los principales casos de uso del sistema.
Se muestran las principales interacciones que hay desde el exterior (votantes, rectorado, partidos pol\'iticos) con el sistema.

Notese que existe una herencia de actores para los casos de uso. Los actores Alumno, Graduado y Profesor heredan del actor Usuario externo.


\begin{itemize}
\bigskip
\item Generando contraseña
\bigskip
\begin{center}
\begin{tabular}{ll}
Actor & Usuario externo \\
\hline
Pre condición & Ninguna \\
\hline
Pos condición & El usuario externo posee usuario y contraseña para el sistema\\
\hline
\end{tabular}
\medskip
\\
Detalle: Se dispondrá de una terminal en la sede que le corresponde a cada usuario externo para que pueda generar por primera vez su contraseña personal. Pudiendo chequear así la identidad del usuario externo.
\end{center}

\bigskip
\item Ingresando al sistema
\bigskip
\begin{center}
\begin{tabular}{ll}
Actor & Usuario externo\\
\hline
Pre condición & El usuario externo posee usuario y contraseña \\
\hline
Pos condición & El usuario externo se encuentra autenticado\\
\hline
\end{tabular}
\medskip
\begin{tabular}{c p{4cm}|p{4cm}}
 & Curso normal & Curso alternativo \\
 1. & El usuario externo indica su nombre de usuario y contraseña &  \\
 2. & El sistema valida los datos ingresados & \\
 3. & Si los datos son correctos el usuario externo recibe una confirmación de que su nombre de usuario y contraseña son correctos & Si los datos son incorrectos el usuario externo recibe un mensaje de error indicando que la combinación de nombre de usuario y contraseña es incorrecta \\
 4. & Se muestra la interfaz de voto electrónico para el usuario & Fin del caso de uso \\
 5. & Fin del caso de uso& \\ 
\end{tabular}
\end{center}


\bigskip
\item Postulandose para las elecciones
\bigskip
\begin{center}
\begin{tabular}{ll}
Actor & Usuario externo \\
\hline
Pre condición & El usuario externo se encuentra logueado \\
\hline
Pos condición & El usuario externo se postula para el claustro correspondiente\\
\hline
\end{tabular}
\medskip
\begin{tabular}{c p{4cm}|p{4cm}}
 & Curso normal & Curso alternativo \\
 1. & El usuario externo se postula para su claustro &  \\
 2. & El sistema valida que se cumplan las restricciones para la elección abierta & \\
 3. & Si se cumplen las restricciones se recibe una confirmación de postulación & Si los datos son incorrectos el usuario recibe un mensaje de error\\
 4. & Fin del caso de uso & Fin del caso de uso \\
\end{tabular}
\end{center}

\bigskip
\item Visualizando cantidatos
\bigskip
\begin{center}
\begin{tabular}{ll}
Actor & Usuario externo\\
\hline
Pre condición & El usuario externo se encuentra auntenticado\\
\hline
Pos condición & El usuario externo visualiza los candidatos\\
\hline
\end{tabular}
\\
\medskip
Detalle: Los usuarios externos pueden visualizar a todos los candidatos que se pueden votar en su claustro.
\end{center}

\bigskip
\item Emitiendo voto
\bigskip
\begin{center}
\begin{tabular}{ll}
Actor & Usuario externo \\
\hline
Pre condición & El usuario externo se encuentra visualizando los candidatos y no voto\\
\hline
Pos condición & Se registra el voto del usuario externo\\
\hline
\end{tabular}
\medskip
\begin{tabular}{c p{4cm}|p{4cm}}
 & Curso normal & Curso alternativo \\
 1. & Se muestra ante el usuario la lista de candidatos de su claustro, con una opción para votar en blanco. &  \\
 2. & El usuario selecciona el candidato de su preferencia o votar en blanco & \\
 3. & Se pide confirmación al usuario mostrando claramente qué opción ha elegido & \\
 4. & El votante confirma su elección & Si el votante no confirma la elección en 30 segundos se descarta la selección del usuario \\
 5. & Se muestra el certificado por pantalla & Fin de caso de uso \\
 6. & Fin de caso de uso & \\ 
\end{tabular}
\end{center}




\bigskip
\item Modificando el idioma en la interfaz de usuario
\bigskip
\begin{center}
\begin{tabular}{ll}
Actor & Usuario externo \\
\hline
Pre condición & El usuario externo se encuentra autenticado \\
\hline
Pos condición & La interfaz se presenta en el idioma elegido y se guarda la preferencia\\
\hline
\end{tabular}
\medskip
        \\
Detalle: El usuario externo tiene la posibilidad de elegir el idioma en el que quiere que la interfaz se presente mediante una elección en un botón desplegable.
\end{center}





\bigskip
\item Agregando elección
\begin{center}
\begin{tabular}{ll}
Actor & Rectorado \\
\hline
Pre condici\'on & Ninguna \\
\hline
Pos condici\'on & Se agrega una nueva elección \\
\hline
\end{tabular}
\medskip
\\
Detalle: En este caso de uso el rectorado agrega una nueva elección seteando el calendario, las restricciones, el claustro y la facultad para la misma.
\end{center}


\bigskip
\item Modificando elección
\begin{center}
\begin{tabular}{ll}
Actor & Rectorado \\
\hline
Pre condici\'on & Ninguna \\
\hline
Pos condici\'on & Se modifica una elección aún no finalizada\\
\hline
\end{tabular}
\medskip
\\
Detalle: En este caso de uso el rectorado modifica el calendario de una elección vigente cambiando los períodos, realizando los chequeos necesarios para mantener la validez de la elección.
\end{center}


\bigskip
\item Reglamentando elecciones
\begin{center}
\begin{tabular}{ll}
Actor & Rectorado \\
\hline
Pre condici\'on & Ninguna \\
\hline
Pos condici\'on & Se cambia el reglamento para las elecciones \\
\hline
\end{tabular}
\medskip
\\
Detalle: En este caso de uso el rectorado modifica las restricciones que tienen las elecciones para la postulación de candidatos, para saber qué personas están habilitadas para votar y para indicar cuántos votos por integrante de cada claustro son aceptados.
\end{center}

\bigskip
\item Auditando votantes
\begin{center}
\begin{tabular}{ll}
Actor & Rectorado \\
\hline
Pre condición & Ninguna\\
\hline
Pos condición & El Rectorado sabe qu\'e votantes sufragaron \\
\hline
\end{tabular}
\medskip
\\
Detalle: En este caso de uso se puede ver cómo el Rectorado puede visualizar los votantes que ya sufragaron en una elección dada.
\end{center}  

\bigskip
\item Auditando votantes
\begin{center}
\begin{tabular}{ll}
Actor & Partido político \\
\hline
Pre condición & Ninguna\\
\hline
Pos condición & El Partido sabe qu\'e votantes sufragaron \\
\hline
\end{tabular}
\medskip
\\
Detalle: Caso similar a la auditación por parte del rectorado.
\end{center} 

\bigskip
\item Viendo resultados
\begin{center}
\begin{tabular}{ll}
Actor & Rectorado \\
\hline
Pre condición & Ninguna\\
\hline
Pos condición & El Rectorado visualiza los resultados de una elección \\
\hline
\end{tabular}
\medskip
\\
Detalle: En este caso de uso se puede ver cómo el Rectorado puede visualizar los resultados de una elección finalizada.
\end{center} 

\bigskip
\item Viendo resultados
\begin{center}
\begin{tabular}{ll}
Actor & Partido político \\
\hline
Pre condición & Ninguna\\
\hline
Pos condición & El Rectorado visualiza los resultados de una elección \\
\hline
\end{tabular}
\medskip
\\
Detalle: En este caso de uso se puede ver cómo el Rectorado puede visualizar los resultados de una elección finalizada.
\end{center} 

\bigskip
\item Fiscalizando votaci\'on
\begin{center}
\begin{tabular}{ll}
Actor & Partido Político \\
\hline
Pre condición & El usuario del partido se encuentra visualizando los resultados de la elección\\
\hline
Pos condición & El Partido Político fiscaliza la elección elegida\\
\hline
\end{tabular}
\medskip
\\
Detalle: Un partido político tiene la potestad de fiscalizar cualquier elección finalizada por demanda.
\end{center}


\bigskip
\item Aceptando acta
\begin{center}
\begin{tabular}{ll}
Actor & Rectorado \\
\hline
Pre condición & Se cerró una elección, se generaron los resultados y se están visualizando\\
\hline
Pos condición & Se genera el acta\\
\hline
\end{tabular}
\medskip
\\
Detalle: Cuando se cierra una elección, se generan los resultados que son enviados a Rectorado. Luego un representate tiene que validar los resultados obtenidos para que se genere el acta pertinente.
\end{center}

\bigskip
\item Resolviendo conflictos
\begin{center}
\begin{tabular}{ll}
Actor & Rectorado \\
\hline
Pre condición & Se cerró una elección, se generaron los resultados y se están visualizando\\
\hline
Pos condición & Se genera un acta con los resultados y una enmienda manual\\
\hline
\end{tabular}
\medskip
        \\
Detalle: Cuando se cierra una elección, se generan los resultados que son enviados a Rectorado. Si estos resultados son conflictivos un representante del Rectorado tiene que agregar una enmienda manualmente que aparecerá junto con los resultados automáticos.
\end{center}

\bigskip
\item Visualizando fechas
\begin{center}
\begin{tabular}{ll}
Actor & Rectorado/Partido Político \\
\hline
Pre condición & Hay una elección abierta\\
\hline
Pos condición & El Rectorado/Partido Político visualiza las fechas de la elección elegida\\
\hline
\end{tabular}
\\
\medskip
Detalle: Tanto el Rectorado como los partidos políticos estan habilitados para visualizar las fechas de las elecciones abiertas.
\end{center}


 

\end{itemize}

\newpage

%Parte opcional en realidad, hay que agregar los gantt y tratar vincular cada modulo del wbs con casos de uso
\section{Plan de proyecto}

\subsection{Iteraciones}


La primera iteración comprenderá principalmente tareas de gestión y capacitación de los desarrolladores. Se tomó esta decisión porque identificamos varios intereses en conflicto y consideramos que es necesario establecer una prioridad entre ellos tan tempranamente como sea posible. Para hacer esto correctamente será necesario en primer lugar capacitar al equipo de desarrollo y de comunicación con los stakeholders en temas de seguridad. Priorizamos este tema en particular por haber sido considerado un atributo de calidad prioritario en el QAW y por la falta de capacitación del equipo en el mismo. 

Una vez realizada la capacitación será necesario identificar arquitecturas alternativas priorizando de manera diferente los distintos intereses en conflicto.

Contando con distintas alternativas arquitectónicas se realizará una serie de reuniones con los distintos stakeholders informándoles por qué los distintos intereses entran en conflicto desde un punto de vista tecnológico y en qué medida beneficia cada posible decisión arquitectónica a la concreción de un objetivo u otro.

Es crucial que la totalidad del equipo tenga conocimientos suficientes de seguridad para garantizar que no sean introducidas vulnerabilidades en módulos aparentemente no relacionados con el tema, por lo que la capacitación de los desarrolladores continuará y se solapará con las reuniones con los stakeholders. 




\subsubsection{Gantt}

\subsection{Work Breakdown Structure}



Utilizando la dinámica sugerida en las clases teóricas, generamos un WBS híbrido. En el primer nivel dividimos el sistema en 5 productos y procesos que capturan las distintas actividades a realizarse para la construcción del sistema completo. Cada uno de los productos y procesos del primer nivel agrupan funcionalidades relacionadas, excepto Actividades de gestión, que se diferencia de los demás por relacionarse con el sistema en su totalidad.

Esta división no muestra todas las dependencias entre productos y procesos, muchas de las cuales terminarán de definirse en el diagrama de Gantt de la iteración que corresponda.

Los cinco elementos del primer nivel serán Interfaz, Sistema Facultad, Comunicación, Actividades de gestión y Fiscalizador.

Interfaz comprende los productos y procesos relacionados con el relevamiento de requerimientos, casos de uso, codificación y prueba de la interfaz. Este producto seguramente podrá ser construído independientemente de los demás.


Sistema Facultad incluye productos y procesos orientados a la construcción del sistema autónomo de cada Facultad, que se encargará de recibir y fiscalizar la totalidad de los votos correspondientes a la misma. Los requerimientos funcionales y atributos de calidad prioritarios definitivos para este producto dependerán de varias tareas de gestión a realizarse en las primeras iteraciones, con el fin de consensuar soluciones de compromiso a los varios intereses que actualmente se encuentran en conflicto, como ser la posibilidad de contar con resultados parciales y un registro de votantes en tiempo real, en contraposición con la idea de garantizar el voto secreto.


Comunicación está conformado por productos y procesos con el fin de proveer de comunicación entre los distintos sistemas de cada Facultad. Este poducto, tal como Sistema Facultad, depende de la definición de varios aspectos de los atributos de calidad y requerimientos funcionales, por lo que dependerá de los procesos de Actividades de gestión.


Actividades de gestión reúne diversos procesos que pueden ser separados en unos pocos ejes. En primer lugar está la capacitación de los stakeholders, explicando por qué los intereses expresados entran en conflicto y cuáles alternativas arquitectónicas permiten priorizar uno u otro y en qué medida.
En segundo lugar habrán gran cantidad de procesos destinados a la validación intermedia del avance de la construcción del sistema, que consistirán en reuniones pequeñas con los principales interesados y algunas reuniones más grandes donde se muestren funcionalidades de interes general.
También habrán actividades de planificación en general, que comprenderá, por ejemplo, la confección de los diagramas de Gantt de cada iteración.
Por último será necesario capacitar al equipo en temas de seguridad, para lograr que sea tenida en cuenta en todos los niveles de desarrollo y elicitación de requerimientos. Esta capacitación será una de las primeras actividades a realizarse.


 


\newpage

%Hay que hacer mas cortos los escenarios, mas directos



\documentclass[a4paper,11pt]{article}

\usepackage[spanish, activeacute]{babel}

\usepackage{color}
\usepackage{graphicx}
\usepackage[utf8x]{inputenc}
\usepackage{fontenc}
\usepackage{listings}


%comando para armar escenario toma 7 parametros, en orden
%texto chamuyyo fuente estimulo entorno artefacto respuesta medicion
\newcommand\escenario[7]{
	\begin{itemize}
		%texto chamuyo
		\item \textit{#1}  
		
		\item \textbf{Fuente:} #2
		\item \textbf{Estímulo:} #3
		\item \textbf{Entorno:} #4
		\item \textbf{Artefacto:} #5
		\item \textbf{Respuesta:} #6
		\item \textbf{Medición:} #7
		
		
	\end{itemize}
}








%\begin{document}

\section{Escenarios}

Tras presenciar el QAW y releer su resumen decidimos plasmar todos los escenarios separados por categorias. Gracias a esto pudimos tomar decisiones arquitectonicas a fin, satisaciendo los atributos de calidad relevados.

% ejemplo de uso
% \escenario{Si la comunicación entre una Terminal de Cobro móvil y el Sistema Central se pierde, o los tiempos de transmisión son prohibitivos, la Terminales de Cobro debe de seguir funcionando en modo offline, de forma transparente al usuario.}{Interno al sistema} {Las transacciones de cobro de pasaje en una terminal de cobro están tardando mas de 1 segundo.}{Operación normal.}{Terminal de Cobro.}{La terminal pasa a modo offline, utilizando la información local almacenada en las tarjetas y logueando  toda operación realizada para que este lista cuando la conexión se restablezca.} {No se requieren acciones adicionales por parte del usuario, y no se ve afectada la performance (<= 1 segundo por transacción).}

\subsection{Disponibilidad}

\begin{enumerate}
 



%Nicolas dijo:
%"Es fundamental que se pueda votar siempre, y en todo momento."


%Info del tp relacionada:

%Es de esperar que haya problemas de conectividad, por lo que se espera que el sistema
%pueda funcionar también en modalidad offline. Bajo esta modalidad, los votos serán
%almacenados de manera temporal y segura en la sede correspondiente, para luego ser
%transmitidos al momento de recuperar la conexión. De la misma manera, esta modalidad
%deberá funcionar para la comunicación entre el rectorado y las facultades.

\item \escenario{Se espera que el sistema
pueda funcionar también en modalidad offline.}{Interno al sistema.}{Hay problemas de conectividad, se cay\'o la conexión entre el sistema y GUI-onlines}{Operación normal}{GUI-offline}{Las GUI-offline detectan que el sistema est\'a offline y habilitan la votación local en modalidad offline}{La GUI-offline detecta que el sistema perdi\'o conectividad y habilita votación offline en menos de 0.3 segundos}

%\textbf{Javier:}
%"Quiere tener comunicacio\'on constante con otras facultades y rectorado"
%\
%Info del tp relacionada:
%\\
%A fin de cumplir con el Estatuto Universitario, el Rectorado mantendrá la potestad de
%vdefinir los requisitos que deben cumplir los electores, junto con el calendario de elecciones
%y otras normas que sirvan para preservar cierta igualdad entre unidades académicas.
%Asimismo, el Rectorado espera contar con información actualizada al instante de todos
%los procesos simultáneos que se desarrollen.

%\textbf{Aunque de lo que dice ac\'a alcanzar\'ia con que todas las facultades se comuniquen con el rectorado nomas, no me alcanza para pedir comunicaci\'on entre facultades}

\medskip
\item \escenario{El Rectorado espera contar con información actualizada al instante de todos
los procesos simultáneos que se desarrollen.}
{Conexi\'on con otra facultad/rectorado}
{Error en la conexi\'on}
{Normal}
{Sistema}
{Utilizar una conexi\'on alternativa}
{El sistema estar\'a conectado nuevamente en menos de 15 segundos}

%\textbf{Javier:}
%"En caso de problemas de conexión, no puede perderse la posibilidad de
%votar en una sede."
%\textbf{Ya esta cubierto por un atributo anterior}



\medskip
%~ Se asemeja a otro de javier
\item \escenario{Algunos no est\'an convencidos de utilizar el voto offline y afirman que rectorado debería garantizar conectividad}
{Votante}
{El usuario emite su voto exitosamente}
{Normal}
{Sistema}
{El voto es registrado correctamente}
{Se garantiza una disponibilidad de 99,99999\%}
\end{enumerate}

\newpage

\subsection{Performance}
\begin{enumerate} 

%Nicolas: "Quiere que el recuento de los votos sea inmediato"

\item \escenario{Al finalizar el acto electoral el recuento de votos deber\'a ser inmediato}{Interno}{Termina el período de una elección}{Normal}{Sistemas en rectorado y sistemas en cada facultad}{Se propagan los resultados al sistema de Rectorado}{El recuento de votos total se hace en 5 segundos y el env\'io de los votos de cada facultad se hace en menos de 1 segundo}
\medskip
%\textbf{Javier:}
%Le preocupa que la comunicación hacia los bunkers de las agrupaciones
%haga muy lenta la conexión (performance/disponiblidad).
%\\
%Info del tp relacionada:
%\\
%Debe publicarse en internet una copia de todo el
%código fuente, más la firma digital de la versión en uso, como así también debe ser
%posible que cada agrupación fiscalice el funcionamiento correcto de la aplicación por
%medio de la auditoría en tiempo real y ejecución paralela de copias del mismo software,
%corriendo en los equipos propios de cada fuerza política.

% \item \escenario{La auditoría se realiza en tiempo real y ejecución paralela.}
% {Votante} %Partido ?
% {Emite un voto}
% {Sobrecargado}
% {Sistema}
% {Se registra el voto y se env\'ia a los fiscalizadores}
% {En menos de 5 segundos los bunkers est\'an informados de que el votante sufragó}


% \medskip
% \item \escenario{Los resultados deben recibirse al mismo tiempo que se van generando para luego ser contrastastados con el escrutinio definitivo}
% {Votante}
% {El usuario emite su voto exitosamente.}
% {Normal}
% {Sistema}
% {El voto es registrado correctamente.}
% {En menos de 1 hora los resultados de votaci\'on son recibidos por todas las agrupaciones políticas.}

\end{enumerate}

\newpage

\subsection{Seguridad}


Nicolas: "Debe bancarse procesos de auditorías y recuentos (seguridad-
auditabilidad)"
\escenario{Debe bancarse procesos de auditorías y recuentos}{Fuente: Pedido de cualquier persona}{Requiere auditar los votos para recontarlos}{Normal}{Sistema}{Le entrega online o en rectorado, una copia en formato digital de los votos emitidos, todo el material estar certificado}{Debe remitir información en menos de una semana}


\escenario{Francisco defiende a ultranza el derecho al voto anónimo}{Atacante con acceso privilegiado al sistema}{El atacante identificar a quién votó un elector en particular}{Normal}{Sistema}{El sistema guarda registro de todos los accesos realizados por el atacante}{En ningún caso es posible, a partir de la información obtenida, asegurar que un votante en particular votó a un cantidato en particular en un tiempo menor a 100 años con las capacidades de cómputo actuales.}



\newpage

\subsection{Modificabilidad}

Nicolas: "Quieren usar el sistema también para otro tipo de elecciones, por ejemplo,
plebiscitos (modificabilidad/flexibilidad)"


\escenario{Quieren usar el sistema también para otro tipo de elecciones, por ejemplo,
plebiscitos}{Rectorado}{De desea confeccionar un plebiscito}{Normal}{Sistema}{Se configura el sistema para que pueda ofrecer participar en un plescbicito sin ninguna cambio negativo al sistema actual}{Se invierte menos del 12,345678 de las horas empleadas para diseñar el sistema original}


\newpage


\subsection{Usabilidad}

\begin{enumerate}

%Adrian: Que se pueda votar desde diversas plataformas, celulares, todo!. (extensibilidad, modificabilidad, funcionalidad, flexiblidad)
\item \escenario
{Que se pueda votar desde diversas plataformas, celulares, todo.}
{Un votante}
{Se intenta votar desde una Tablet conectada con 4G por internet.}
{Normal}
{Sistema web}
{El sistema permite votar siendo transparente la plataforma para \'este.}
{El sistema funciona correctamente en los 5 dispositivos m\'as utilizados.}


\item  \escenario{Se quiere que el sistema sea sencillo de usar, para que no se confunda al electorado al momento de emitir el voto}{Usuario}{El usuario ingresa el voto para el candidato que eligió}{Operación normal}{Sistema}{Se registra el voto del usuario}{El 100 \% de los usuarios efectivamente votó al usuario que cree haber votado.}

\medskip




\item  \escenario
{La interfaz debe ser simple, para que no consuma mucho tiempo del usuario}
{Usuario}
{El usuario ingresa el voto para el candidato que eligió}
{Operación normal}
{Sistema}
{Se registra el voto del usuario}
{El 99\% de los usarios logran emitir su voto en dos minutos.}
\medskip



\medskip
\item   \escenario{Usable en otros idiomas!, fácil de configurar el idioma}
{Un votante}
{Un votante que domina mal el espa\~nol ingresa el sitio para votar y desea cambiar el idioma a ingl\'es}
{Operación Normal}
{Sistema}
{El sistema permite elegir el idioma y cambiar sus contenidos al idioma correspondiente.}
{El usuario encuentra c\'omo cambiar el idioma en menos de 15 segundos.}
\medskip


\end{enumerate}


%\end{document}


\newpage




Presentaremos la arquitectura explicando cómo se brindan las funcionalidades requeridas, mencionando diseños alternativos y algunos aspectos relevantes de la comunicación. Por último se comentará cómo se logra cada atributo de calidad a partir de las tácticas utilizadas y la estructura descripta. 

\section{Presentación de la arquitectura}

El uso más habitual que se le dará al sistema será el de votar, por lo que comenzaremos explicando cómo se logra esta funcionalidad con la arquitectura propuesta, enfocándonos principalmente en el diagrama de componentes y conectores.

Para votar, un usuario deberá conectarse por Internet al sitio designado por su Facultad a tal efecto. El browser que utiliza el usuario se comunicará con el componente Vox de la Facultad. El browser mostrará la opción de autenticarse frente al sistema mediante un nombre de usuario y contraseña, previamente generado por el sistema.

Esta información será enviada por una conexión encriptada y confiable al componente Vox de la Facultad que validará dicha información y retornará al browser un certificado de autenticación que será utilizado subsecuentemente para garantizar que la comunicación cumpla con las características de seguridad deseadas.

Tras autenticar exitosamente al usuario, el browser mostrará los candidatos disponibles, también recibidos del componente Vox de la Facultad. Si el usuario concretara su voto luego recibiría un certificado del mismo, generado por el componente Vox de la Facultad.

Con esta breve introducción se aprecia que será responsabilidad del componente Vox de la Facultad recibir combinaciones de nombres de usuario y contraseña y poder verificar su validez, también contar con la información de los candidatos y por último la capacidad de generar certificados de los votos recibidos, funcionalidades que veremos a continuación.


El componente Vox de la Facultad resultó polémico debido a que gran parte de la funcionalidad será brindada por el mismo y existen varios intereses que entran en conflicto al momento de elegir una arquitectura en particular. Una primera propuesta, como puede verse en el diagrama % TODO referenciar al diagrama
, se apoya fuertemente en la premisa de que los subcomponentes de administración de votos (VOX-Core) y portal online se ejecutarían en servidores diferentes y que podrían existir problemas de conexión entre los mismos, para lo que sería necesario proveer algún mecanismo que permitiera ejercer el derecho a voto aún sin poder contar con las funcionalidades de administración de votos. Esta alternativa no terminó de detallarse porque resultaron evidentes varias complicaciones que ponían en riesgo la seguridad del sistema y algunos criterios de usabilidad. Concretamente los defectos que encontramos en esta opción se relacionan con la necesidad de guardar los votos asociados a sus votantes en un repositorio externo a la facultad hasta tanto se recupere la conectividad con la facultad y que durante ese tiempo de desconexión el usuario no contaría con un certificado de su voto y, en el mejor de los casos, recibiría un mensaje informándole de los problemas de conexión y dejándolo con gran incertidumbre respecto del efectivo cómputo del mismo.


Considerando la prioridad asignada a los atributos de calidad en el QAW realizado y siendo confiabilidad menos prioritario que usabilidad y que seguridad optamos por favorecer una estructura diferente, que aporta menos garantías de confiabilidad pero mejora sensiblemente los aspectos antes mencionados de usabilidad y seguridad.

Se puede apreciar en el diagrama % TODO referenciar al diagrama
que en la propuesta elegida se elimina la necesidad de mantener un repositorio de votos no procesados, como ocurría con la anterior opción. Sin embargo, para poder mantener la posibilidad de votar en todo momento, será necesario que el componente del portal esté siendo ejecutado en la Facultad misma. Varias sutilezas con respecto a este tema serán discutadas más adelante, al momento de referirse a las tácticas y soluciones encontradas para los atributos de calidad.


La estructura propuesta permite que el conteo de votos se lleve a cabo en cada Facultad independientemente de las demás, pero esta descentralización, junto con la necesidad de fiscalizar los resultados realizando un recuento completo, conlleva un costo elevado en cuanto a la complejidad de la conexión entre las distintas facultades y los componentes correspondientes a la fiscalización realizada por cada agrupación política y por el rectorado.

Para apreciar el funcionamiento del componente Vox-Facultad, seguiremos los pasos a realizarse para el proceso de un voto, asumiendo que la autenticación ya fue realizada. Una vez arribado el voto, a través del componente portal, el mismo es transmitido dentro de un mensaje al administrador de votos, junto con el nombre de usuario del sufragante. Dentro de VOX-Core, el componente del comunicador desencripta el mensaje y lo envía al auditor que verifica que el usuario no haya votado mediante un acceso al repositorio de gente que votó, verifica el cumplimiento por parte del usuario de las restricciones impuestas por rectorado, accediendo al administrador de datos de usuarios % TODO agregar algo por el estilo
y verifica la validez del candidato accediendo al repositorio. % TODO agregar repositorio a los diagramas
De haberse realizado todos los anteriores pasos con éxito, se registra un voto para el candidato correspondiente, se envía el voto al publicador y se genera, a partir del sistema externo HardToBreak, un certificado de la emisión del voto, que se retornado al comunicador. Es importante destacar que el voto en sí no contiene información del votante, pero podría contener algún tipo de hash o certificado, que mejore la auditabilidad del sistema. 
 
Para satisfacer las necesidades de fiscalización expresadas, tras procesarse el voto, el mismo es introducido en una cola de envío, % TODO agregar cola al diagrama
para luego ser enviado a través de un conector multicast seguro y con varias propiedades adicionales que serán oportunamente mencionadas.

El proceso que seguirá una vez recibido por los componentes fiscalizadores de las agrupaciones políticas y rectorado consiste en registrar el voto para el candidato correspondiente. La frecuencia con que efectivamente se envían los votos deberá determinarse con mayor participación de los diferentes stakeholders, ya que el envío inmediato de los votos junto con la posibilidad de acceso en todo momento a la información de los votantes que ya emitieron su voto hace imposible la tarea de mantener el voto secreto. El funcionamiento de la cola de envíos queda por el momento indefinido, pero puede ser tan simple como enviar inmediatamente los votos encolados hasta un esquema más complejo, que contenga restricciones sobre la mínima cantidad de votos acumulados o la mínima cantidad de candidatos representados en los mismos, por ejemplo.

En cada componente fiscalizador se recibe el conteo de votos de todas las facultades y se calculan los resultados totales. Luego mediante la interfaz por browser del fiscalizador es posible filtrar que porcentajes se muestran.

En el componente fiscalizador es posible mediante la interfaz por browser filtrar que 

El componente dedicado a las tareas administrativas del rectorado permite, mediante una comoda interfaz por browser, administrar las reglas de los comiicios y enviarlas en simultaneo a todas las facultades. La interfaz también da un marco para ingresar los candidatos que luego se propagaran a los repositorios de los componentes Vox-Facultad.
\\
Desde el sistema de rectorado se genera, antes de iniciar los comicios, las contraseñas para todos los usuarios y se les envia al correo electrónico que tengan registrado en el padrón. Luego a cada facultad se le envia la información necesaria para registrar el loggueo de sus votantes.


% TODO seguir explicando 



\subsection{Obtención de los atributos de calidad}


Dada la necesidad de asegurar un nivel elevado de seguridad en todos los niveles, se utilizaron varias tácticas con dicho objetivo. En todas las comunicaciones se utiliza encripción, suma de verificación y un esquema de firmas para garantizas no repudio. La conexión de las interfaces gráficas para los votantes con el resto del sistema utilizan autenticación.


Dado que los repositorios de usuarios que ya votaron y de cantidad de votos son críticos para lograr auditabilidad, los mismos se encuentran duplicados en diversas máquinas.

Para mejorar la usabilidad se separó la interfaz gráfica del resto de los componentes, de manera que pueda evolucionar por separado, priorizando criterios de facilidad de uso y de aprendizaje.


 


\newpage

%Dividir mitigacion y contingencia
\section{An\'alisis de Riesgos}

En esta secci\'on se analizar\'an los riesgos pertinentes al proyecto.

El objetivo de este an\'alisis es poder crear un plan de mitigaci\'on y contingencia para poder controlar los riesgos que puedan afectar al proceso del desarrollo del producto, de manera que se pueda minimizar el impacto que los posibles riesgos produzcan en caso de manifestarse.

Los pasos m\'as importantes en este an\'alisis son:

\begin{itemize}
\item Identificar los riesgos que puedan manifestarse.
\item Analizarlos individualmente.
\item Documentarlos. En este paso se utilizar\'a la representaci\'on de Glutch para especificarlos y luego se utilizar\'a la Matriz de Magnitudes del SEI para analizar el nivel del riesgo en base a su probabilidad de ocurrencia y a su severidad.
\item Generar un plan de mitigaci\'on y contingencia para cada uno de los riesgos identificados.
\end{itemize}


\subsection{Riesgos}


A continuaci\'on se muestran los riesgos m\'as relevantes identificados por el grupo, especificados y analizados como se explicit\'o anteriormente.

\begin{itemize}
\item Dificultades para funcionar co-operativamente con los sistemas de los partidos pol\'iticos: Dada la heterogeneidad natural en los sistemas de los partidos pol\'iticos, y que no hay experencias previas de cooperativiadad con estos sistemas, es posible que se encuentren dificultades para garantizar la comunicabilidad a todos los fiscalizadores pertinentes. \begin{itemize}                                                                                                                                                                                                                                                                                                                                                                            									   \item Probabilidad: Probable.
                                                                           \item Severidad: Cr\'itica.
									   \item Nivel: Alto								\end{itemize}

\bigskip

\item Problemas con las tecnolog\'ias a utilizar: Dado los atributos de calidad relevados, se necesitar\'a trabajar sobre muy diversas plataformas, al desconocer varias de estas plataformas de \'ultima generaci\'on es posible que se retrasen los tiempos de producci\'on al necesitar invertir tiempo en investigaci\'on de los diferentes dispositivos soportados por el software. 
\begin{itemize}
\item Probabilidad: Muy probable.
\item Severidad: Media.
\item Nivel: Alto.
\end{itemize}

\bigskip

\item 

\end{itemize}

- reduccion de personal
- cambios en los requerimientos
- capacidad de los servidores
- disponibilidad de los servidores
- recorte de presupuesto

\newpage
\section{Comparaci\'on y conclusiones}

En primer lugar es conveniente destacar que la comparaci\'on se realiza unicamente habiendo hecho la planificaci\'on, pero sin haber ejecutado ninguna de las iteraciones definidas. Por lo cual la comparaci\'on entre los m\'etodos estar\'a limitada.

Recordemos como fue el primer trabajo pr\'actico. En dicho trabajo no se planific\'o todo el proceso de desarrollo, sino que se ejecuto solamente un sprint, que se diferencia de una iteraci\'on en UP en varios aspectos entre los cuales se encuentra que UP define dependencias entre las distintas tareas y las diferentes duraciones para cada una de las mismas; mientras que en m\'etodos \'agiles todas las tareas a relizarse deber\'ian ser independientes y poder ejecutarse en la duraci\'on de la iteraci\'on.


A continuaci\'on compararemos varios aspectos de ambos m\'etodos:

\begin{itemize}
\item Tiempo de planificaci\'on: En UP el tiempo que nos llev\'o realizar toda la planificaci\'on, fue el mismo que en m\'etodos \'agiles necesitamos para tener algo \emph{potentially shippable}.
\item Detalle de la planificaci\'on: Al momento de iniciar la primera iteraci\'on en UP, contamos con un cronograma definido, planes de mitigaci\'on y contingencia de riesgos, y un an\'alisis de las dependencias entre las tareas. Por otro lado, en los m\'etodos \'agiles, solamente se conoce un pool de tareas a realizar y un tiempo fijo para todas ellas.
\item Capacitaci\'on de Stakeholders: Si bien las metodolog\'ias difieren, en ambos casos es necesario capacitar a los Stakeholders. En el caso de UP, se deber\'a explicar como realizar un QAW y sus objetivos; mientras que en el caso de metodolog\'ias \'agiles, se requiere una mayor participaci\'on y comprensi\'on de la metodolog\'ia. 
\item Validaci\'on contra Stakeholders: En UP, esta validaci\'on se hace de forma integral con toda la arquitectura y adem\'as se validan dependencias entre distintas actividades del proceso de desarrollo; en m\'etodos \'agiles, solo se validan los incrementos, y al inicio los requerimientos.
\end{itemize}


Cada uno de los puntos presentados tienen diversas ventajas y desventajas tanto de un punto de vista de desarrollo, como desde la facilidad de comunicaci\'on con el usuario, para ambas metodolog\'ias presentadas. Por ejemplo, en cuanto al tiempo de planificaci\'on, no es claro que uno de los m\'etodos sea mejor que el otro, dependiendo el tipo de proyecto que estemos afrontando puede ser conveniente y hasta necesaria una elecci\'on de alguno de los dos m\'etodos por sobre el otro. En el caso de UP, se tom\'o mucho tiempo y no se lleg\'o a tener ninguna implementaci\'on preliminar; sin embargo, se logr\'o una buena planificaci\'on integral, y un nivel de detalle que permite una comunicaci\'on mucho m\'as flu\'ida con los Stakeholders, ya que se cuenta con estimaciones de plazos y dependencias entre las actividades. Muchas veces, es necesario de antemano, poder validar toda la arquitectura para asegurarse que no pueden aparecer problemas en etapas posteriores. Por otro lado, en Scrum s\'i se logr\'o una primera versi\'on funcional del producto en un tiempo bastante acotado, que en muchos casos podr\'ia ser prioritario.

Por \'ultimo, nos gustar\'ia mencionar que al momento de elegir la metodolog\'ia no basta con una simple comparaci\'on entre dos proyectos, sino que es una decisi\'on que tiene muchas aristas.  Por un lado, es una decisi\'on claramente sesgada por la experiencia laboral del responsable. Por otro lado, existen varios criterios consensuados para determinar la conveniencia de una metodolog\'ia \'agil o una m\'as tradicional. Una posible gu\'ia para tener en cuenta es el trabajo de Steve McConnell \emph{Right-Sizing Agile Development}, que incluye un diagrama en donde determina en cuanto a las prioridades de diferentes aspectos, qu\'e metodolog\'ia es m\'as conveniente.



\end{document}
