\section{Comparaci\'on y conclusiones}

En primer lugar es conveniente destacar que la comparaci\'on se realiza unicamente habiendo hecho la planificaci\'on, pero sin haber ejecutado ninguna de las iteraciones definidas. Por lo cual la comparaci\'on entre los m\'etodos estar\'a limitada.

Recordemos como fue el primer trabajo pr\'actico. En dicho trabajo no se planific\'o todo el proceso de desarrollo, sino que se ejecuto solamente un sprint, que se diferencia de una iteraci\'on en UP en varios aspectos entre los cuales se encuentra que UP define dependencias entre las distintas tareas y las diferentes duraciones para cada una de las mismas; mientras que en m\'etodos \'agiles todas las tareas a relizarse deber\'ian ser independientes y poder ejecutarse en la duraci\'on de la iteraci\'on.


A continuaci\'on compararemos varios aspectos de ambos m\'etodos:

\begin{itemize}
\item Tiempo de planificaci\'on: En UP el tiempo que nos llev\'o realizar toda la planificaci\'on, fue el mismo que en m\'etodos \'agiles necesitamos para tener algo \emph{potentially shippable}.
\item Detalle de la planificaci\'on: Al momento de iniciar la primera iteraci\'on en UP, contamos con un cronograma definido, planes de mitigaci\'on y contingencia de riesgos, y un an\'alisis de las dependencias entre las tareas. Por otro lado, en los m\'etodos \'agiles, solamente se conoce un pool de tareas a realizar y un tiempo fijo para todas ellas.
\item Capacitaci\'on de Stakeholders: Si bien las metodolog\'ias difieren, en ambos casos es necesario capacitar a los Stakeholders. En el caso de UP, se deber\'a explicar como realizar un QAW y sus objetivos; mientras que en el caso de metodolog\'ias \'agiles, se requiere una mayor participaci\'on y comprensi\'on de la metodolog\'ia. 
\item Validaci\'on contra Stakeholders: En UP, esta validaci\'on se hace de forma integral con toda la arquitectura y adem\'as se validan dependencias entre distintas actividades del proceso de desarrollo; en m\'etodos \'agiles, solo se validan los incrementos, y al inicio los requerimientos.
\end{itemize}


Cada uno de los puntos presentados tienen diversas ventajas y desventajas tanto de un punto de vista de desarrollo, como desde la facilidad de comunicaci\'on con el usuario, para ambas metodolog\'ias presentadas. Por ejemplo, en cuanto al tiempo de planificaci\'on, no es claro que uno de los m\'etodos sea mejor que el otro, dependiendo el tipo de proyecto que estemos afrontando puede ser conveniente y hasta necesaria una elecci\'on de alguno de los dos m\'etodos por sobre el otro. En el caso de UP, se tom\'o mucho tiempo y no se lleg\'o a tener ninguna implementaci\'on preliminar; sin embargo, se logr\'o una buena planificaci\'on integral, y un nivel de detalle que permite una comunicaci\'on mucho m\'as flu\'ida con los Stakeholders, ya que se cuenta con estimaciones de plazos y dependencias entre las actividades. Muchas veces, es necesario de antemano, poder validar toda la arquitectura para asegurarse que no pueden aparecer problemas en etapas posteriores. Por otro lado, en Scrum s\'i se logr\'o una primera versi\'on funcional del producto en un tiempo bastante acotado, que en muchos casos podr\'ia ser prioritario.

Por \'ultimo, nos gustar\'ia mencionar que al momento de elegir la metodolog\'ia no basta con una simple comparaci\'on entre dos proyectos, sino que es una decisi\'on que tiene muchas aristas.  Por un lado, es una decisi\'on claramente sesgada por la experiencia laboral del responsable. Por otro lado, existen varios criterios consensuados para determinar la conveniencia de una metodolog\'ia \'agil o una m\'as tradicional. Una posible gu\'ia para tener en cuenta es el trabajo de Steve McConnell \emph{Right-Sizing Agile Development}, que incluye un diagrama en donde determina en cuanto a las prioridades de diferentes aspectos, qu\'e metodolog\'ia es m\'as conveniente.
