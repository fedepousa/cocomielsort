\section{Plan de proyecto}

\subsection{Iteraciones}


La primera iteración comprenderá principalmente tareas de gestión y capacitación de los desarrolladores. Se tomó esta decisión porque identificamos varios intereses en conflicto y consideramos que es necesario establecer una prioridad entre ellos tan tempranamente como sea posible. Para hacer esto correctamente será necesario en primer lugar capacitar al equipo de desarrollo y de comunicación con los stakeholders en temas de seguridad. Priorizamos este tema en particular por haber sido considerado un atributo de calidad prioritario en el QAW y por la falta de capacitación del equipo en el mismo. 

Una vez realizada la capacitación será necesario identificar arquitecturas alternativas priorizando de manera diferente los distintos intereses en conflicto.

Contando con distintas alternativas arquitectónicas se realizará una serie de reuniones con los distintos stakeholders informándoles por qué los distintos intereses entran en conflicto desde un punto de vista tecnológico y en qué medida beneficia cada posible decisión arquitectónica a la concreción de un objetivo u otro.

Es crucial que la totalidad del equipo tenga conocimientos suficientes de seguridad para garantizar que no sean introducidas vulnerabilidades en módulos aparentemente no relacionados con el tema, por lo que la capacitación de los desarrolladores continuará y se solapará con las reuniones con los stakeholders. 




\subsubsection{Gantt}

\subsection{Work Breakdown Structure}



Utilizando la dinámica sugerida en las clases teóricas, utilizamos un WBS híbrido. En el primer nivel dividimos el sistema en 5 productos y procesos que capturan las distintas actividades a realizarse para la construcción del sistema completo.

Esta división no muestra todas las dependencias entre productos y procesos, muchas de las cuales terminarán de definirse en el diagrama de Gantt de la iteración que corresponda.

Los cinco elementos del primer nivel serán Interfaz, Sistema Facultad, Comunicación, Actividades de gestión y Fiscalizador.

Interfaz comprende los productos y procesos relacionados con el relevamiento de requerimientos, casos de uso, codificación y prueba de la interfaz. Este producto seguramente podrá ser construído independientemente de los demás.


Sistema Facultad incluye productos y procesos orientados a la construcción del sistema autónomo de cada Facultad, que se encargará de recibir y fiscalizar la totalidad de los votos correspondientes a la misma. Los requerimientos funcionales y atributos de calidad prioritarios definitivos para este producto dependerán de varias tareas de gestión a realizarse en las primeras iteraciones, con el fin de consensuar soluciones de compromiso a los varios intereses que actualmente se encuentran en conflicto, como ser la posibilidad de contar con resultados parciales y un registro de votantes en tiempo real, en contraposición con la idea de garantizar el voto secreto.


Comunicación está conformado por productos y procesos con el fin de proveer de comunicación entre los distintos sistemas de cada Facultad. Este poducto, tal como Sistema Facultad, depende de la definición de varios aspectos de los atributos de calidad y requerimientos funcionales, por lo que dependerá de los procesos de Actividades de gestión.


Actividades de gestión reúne diversos procesos que pueden ser separados en unos pocos ejes. En primer lugar está la capacitación de los stakeholders, explicando por qué los intereses expresados entran en conflicto y cuáles alternativas arquitectónicas permiten priorizar uno u otro y en qué medida.
En segundo lugar habrán gran cantidad de procesos destinados a la validación intermedia del avance de la construcción del sistema, que consistirán en reuniones pequeñas con los principales interesados y algunas reuniones más grandes donde se muestren funcionalidades de interes general.
También habrán actividades de planificación en general, que comprenderá, por ejemplo, la confección de los diagramas de Gantt de cada iteración.
Por último será necesario capacitar al equipo en temas de seguridad, para lograr que sea tenida en cuenta en todos los niveles de desarrollo y elicitación de requerimientos. Esta capacitación será una de las primeras actividades a realizarse.


 
