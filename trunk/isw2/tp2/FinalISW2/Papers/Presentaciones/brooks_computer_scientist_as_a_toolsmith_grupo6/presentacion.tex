\documentclass{beamer}
\usepackage[spanish]{babel} % Para separar correctamente las palabras
\usepackage[utf8]{inputenc}
\usepackage{graphicx}
% This is the file main.tex
\usetheme{Warsaw}
\setbeamertemplate{blocks}[rounded][shadow=true] 
\setbeamersize{text margin left=4mm} 
\setbeamersize{text margin right=4mm} 

%\includegraphics[height=0.2\textheight]{unne.jpg} \hspace*{7.3cm}

\title{The Computer Scientist as Toolsmith II}
\author{Mariano Bianchi - Pablo Brusco - Pablo Echevarria }
\date{7 de nombriembre 2011}


\begin{document}

 \maketitle 
  

  \section{Sobre el autor: Frederick P. Brooks. Jr.}
    \begin{frame}
     \frametitle{Frederick P. Brooks. Jr.}
	\begin{itemize}
	\item Doctor en matemática aplicada por la Universidad de Harvard
	\item Trabajó en la arquitectura del IBM 7030 ( máquina usada en Los Álamos, costo u\$ 10 M )
	\item Dirigió el desarrollo del sistema operativo OS/360
	\item Autor de la frase: ``Añadir personal a un proyecto retrasado lo retrasará aún más'' ( conocida como la \textbf{ley de Brooks} )
 	\item Escribió el art\'iculo ``No Silver Bullet''
	\item \textbf{Premio Turing 1999}
	\item Brooks es cristiano practicante, muy activo en la comunidad evangélica. ( se ve claramente en el art\'iculo )
	\end{itemize}
	{\small Fuente: wikipedia}
     \end{frame}     

  \section{Ideas principales}
    \begin{frame}
     \frametitle{Ideas principales}
	\begin{block}{}
	    \begin{center}
	      \textbf{ The Computer Scientist as Toolsmith} \\ 
		significa: \\
	      \textbf{ El Científico en Computación es un creador de herramientas }
	    \end{center}	    
	\end{block}
      \begin{itemize}
	\item Problema con el nombre ``Ciencias de la computación''
	\item La sana evolución de la IA
	\item Un ``toolsmith'' como colaborador
	\item Todo es showbusiness
      \end{itemize}
    \end{frame}

  \subsection{Problema con el nombre ``Ciencias de la computación''}
    \begin{frame}
      \frametitle{Problema con el nombre ``Ciencias de la computación''}
      \begin{itemize}
	\item ¿De qué se ocupa una ciencia? De descubrir hechos y leyes, es decir, observar y clasificar hechos para formular leyes generales verificables.
	\item ¿Qué es nuestra disciplina? La ve como una ingeniería sintética, cuyos productos son intangibles. \\
	Cita a H. Zemanek: ciencias de la computación es una ingeniería - el sentido clásico - pero de objetos abstractos
  \end{itemize}    
\end{frame}

\subsubsection{Consecuencias de un nombre errado}
    \begin{frame}
      \frametitle{Consecuencias de un nombre errado}
      \begin{itemize}
	\item Conciderarnos cientificos implica que
	\begin{itemize}
		\item  la invenci\'on de algoritmos y lenguajes son un fin en s\'i mismo. Pero con diferencia a otras ciencias, la novedad no tiene merito.
		\item  olvidamos a los usuarios y a sus problemas reales.
		\item los jovenes se alejan de atacar problemas reales a medida que la disciplina los conduce a abstraer.
	\end{itemize}	
	\item Reconocer que somos Toolshmiths, dar\'ia cr\'edito a nuestro trabajo a trav\'es de su costo y utilidad.

  \end{itemize}    
\end{frame}

\subsubsection{La palabra ``Computaci\'on''  esta bien}
    \begin{frame}
      \frametitle{La palabra ``Computaci\'on''  esta bien}
      \begin{itemize}
	\item La computadora permite al software, manejar un mundo de una complejidad que no era accesible mediante t\'ecnicas manuales.
	\item El dominio de la computaci\'on es este mundo.
	\item Los problemas importantes se caracterizan por tener complejidad arbitraria.
	\item Complejo,  nos aleja de los matem\'aticos
	\item Arbitraria, nos aleja del resto de las ciencias.  
  \end{itemize}    
\end{frame}



  \subsection{La sana evolución de la IA}
    \begin{frame}
      \frametitle{La sana evolución de la IA }
	\begin{block}{}
	    \begin{center}
	    Torre de Babel $\rightsquigarrow$ Amplificador del cerebro humano 
	    \end{center}	    
	\end{block}

	


 
	
      \begin{itemize}
	\item AI $\approx$ Torre de Babel
	\item Se invirtió mucho tiempo y dinero en su desarrollo
	\item Se planteó como objetivo imitar al cerebro humano
	\item Pero no se lograron grandes avances en el área, esta experiencia llevó a entender que era un problema mayor, entonces el problema mutó:
	\begin{itemize}
	  \item Antes: AI = Imitador del cerebro humano
	  \item Luego: IA = Amplificador del cerebro humano
	\end{itemize}
      \end{itemize}
    \end{frame}


  \subsection{Un ``toolsmith'' como colaborador}
    \begin{frame}
      \frametitle{Un ``toolsmith'' como colaborador}
	\begin{itemize}
       	    \item Interación con profesionales de otras disiplinas visto como una simbiosis. Visto como una experiencia enriquecedora
	    \item Ventajas:
	    \begin{itemize}
	    	\item Acercamiento a problemas reales.
	    	\item Vision honesta de \'exito o fracaso.
	    	\item Problema entero (tomando en cuenta casos patol\'ogicos)
	    	\item Enfrentarse a estos problemas hace crecer el conocimiento en la computaci\'on.
	    \end{itemize}
	    \item Pero colaborar tiene su costo
	  \end{itemize}
     \end{frame}

  \subsection{Todo es showbusiness}
    \begin{frame}
      \frametitle{Todo es showbusiness}
	\begin{itemize}
       	    \item Reflexiona sobre las connotaciones negativas de la Television, la cual es escencialmente pasiva y no-social. Y como se vio antes la interacción social es sumamente valiosa
	    \item Cuestiona el medio como método de validar un trabajo. ¿Famoso = Valioso/Válido?
	    \item Afirma que el poder de lo visual puede darnos un medio creativo para el desarrollo de aplicaciones completamente nuevo
	  \end{itemize}
     \end{frame}

    \section{Conclusiones}
      \begin{frame}
            \frametitle{Conclusiones}

       \begin{itemize}
       \item Pensar en los usuarios. 
	  \item Colaborar con otras diciplinas.
	  \item Ser buen científico, en el sentido de cuestionar los términos con los que se nombran a las cosas.
	  \item Cuidado con la fama, ser honesto con lo que uno desarrolla y crea.
	\end{itemize}  
    \end{frame}    

     \begin{frame}
       \begin{center}
 	Gracias :)
 	
 	\vspace{2cm}

 	Preguntas?
       \end{center}
     \end{frame}    
\end{document}

