\documentclass[a4paper,11pt]{article}

\usepackage[spanish, activeacute]{babel}

\usepackage{color}
\usepackage{graphicx}
\usepackage[utf8x]{inputenc}
\usepackage{fontenc}
\usepackage{listings}


%comando para armar escenario toma 7 parametros, en orden
%texto chamuyyo fuente estimulo entorno artefacto respuesta medicion
\newcommand\escenario[7]{
	\begin{itemize}
		%texto chamuyo
		\item \textit{#1}  
		
		\item \textbf{Fuente:} #2
		\item \textbf{Estímulo:} #3
		\item \textbf{Entorno:} #4
		\item \textbf{Artefacto:} #5
		\item \textbf{Respuesta:} #6
		\item \textbf{Medición:} #7
		
		
	\end{itemize}
}








\begin{document}

\section{Escenarios}


% ejemplo de uso
% \escenario{Si la comunicación entre una Terminal de Cobro móvil y el Sistema Central se pierde, o los tiempos de transmisión son prohibitivos, la Terminales de Cobro debe de seguir funcionando en modo offline, de forma transparente al usuario.}{Interno al sistema} {Las transacciones de cobro de pasaje en una terminal de cobro están tardando mas de 1 segundo.}{Operación normal.}{Terminal de Cobro.}{La terminal pasa a modo offline, utilizando la información local almacenada en las tarjetas y logueando  toda operación realizada para que este lista cuando la conexión se restablezca.} {No se requieren acciones adicionales por parte del usuario, y no se ve afectada la performance (<= 1 segundo por transacción).}

\subsection{Disponibilidad}

\begin{enumerate}
 



%Nicolas dijo:
%"Es fundamental que se pueda votar siempre, y en todo momento."


%Info del tp relacionada:

%Es de esperar que haya problemas de conectividad, por lo que se espera que el sistema
%pueda funcionar también en modalidad offline. Bajo esta modalidad, los votos serán
%almacenados de manera temporal y segura en la sede correspondiente, para luego ser
%transmitidos al momento de recuperar la conexión. De la misma manera, esta modalidad
%deberá funcionar para la comunicación entre el rectorado y las facultades.

\item \escenario{Se espera que el sistema
pueda funcionar también en modalidad offline.}{Interno al sistema.}{Hay problemas de conectividad, se cay\'o la conexión entre el sistema y GUI-onlines}{Operación normal}{GUI-offline}{Las GUI-offline detectan que el sistema est\'a offline y habilitan la votación local en modalidad offline}{La GUI-offline detecta que el sistema perdi\'o conectividad y habilita votación offline en menos de 0.3 segundos}

%\textbf{Javier:}
%"Quiere tener comunicacio\'on constante con otras facultades y rectorado"
%\
%Info del tp relacionada:
%\\
%A fin de cumplir con el Estatuto Universitario, el Rectorado mantendrá la potestad de
%vdefinir los requisitos que deben cumplir los electores, junto con el calendario de elecciones
%y otras normas que sirvan para preservar cierta igualdad entre unidades académicas.
%Asimismo, el Rectorado espera contar con información actualizada al instante de todos
%los procesos simultáneos que se desarrollen.

%\textbf{Aunque de lo que dice ac\'a alcanzar\'ia con que todas las facultades se comuniquen con el rectorado nomas, no me alcanza para pedir comunicaci\'on entre facultades}

\medskip
\item \escenario{El Rectorado espera contar con información actualizada al instante de todos
los procesos simultáneos que se desarrollen.}
{Conexi\'on con otra facultad/rectorado}
{Error en la conexi\'on}
{Normal}
{Sistema}
{Utilizar una conexi\'on alternativa}
{El sistema estar\'a conectado nuevamente en menos de 15 segundos}

%\textbf{Javier:}
%"En caso de problemas de conexión, no puede perderse la posibilidad de
%votar en una sede."
%\textbf{Ya esta cubierto por un atributo anterior}



\medskip
%~ Se asemeja a otro de javier
\item \escenario{Algunos no est\'an convencidos de utilizar el voto offline y afirman que rectorado debería garantizar conectividad}
{Votante}
{El usuario emite su voto exitosamente}
{Normal}
{Sistema}
{El voto es registrado correctamente}
{Se garantiza una disponibilidad de 99,99999\%}
\end{enumerate}



\end{document}