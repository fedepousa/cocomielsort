\documentclass[a4paper,10pt]{article}
\usepackage{graphicx}
\usepackage{verbatim}
\usepackage{subfig}
\usepackage{float}
\usepackage[spanish]{babel}   %ver bien como es
\usepackage[utf8]{inputenc}

\begin{document}

\tableofcontents

\newpage


\section*{Introducci\'on}
\addcontentsline{toc}{section}{Introducci\'on}

Cosas que dijo la catedra:

acá debería figurar todo lo necesario como para que un
lector no iniciado en el tema entienda el propósito general del
sistema que van a describir. Puede citar algunos fragmentos del
enunciado si lo consideran necesario, pero NO DEBE SER una copia del
enunciado.

\section*{Presunciones}
\addcontentsline{toc}{section}{Presunciones}

Cosas que dijo la catedra:

acá deberían listar aquellas cuestiones que asumieron
por encima del enunciado. Estas cuestiones pueden provenir de alguna
consulta con docentes. También pueden provenir de alguna especulación
o interpretación que el grupo hizo.
\noindent
\textbf{Tiempo de pedido:} asumimos que dos clientes no puede hacer un pedido de forma simultánea. Esta presunción permite simplificar el cumplimiento del objetivo relacionado a no dejar que se pueda cancelar un pedido a un cliente luego de que fue hecho, dado que establece un claro orden entre ellos. \\
\textbf{Tiempo de pedido a distancia:} asumimos que, ante un pedido a `distancia' (es decir, de un local a otro), un cliente pierde el interés en buscar dicho pedido luego de un \textit{time out} pautado de antemano. Esta presunción nos ayuda a poder cumplir, por un lado, el objetivo relacionado a permitir dicha compra `a distancia' y, por otro, a poder mejorar el manejo de stock al cancelar el pedido luego de pasado el \textit{time out}, y de esta forma recuperar el stock reservado para éste. Esto se logra sin violar los objetivos relacionados al funcionamiento de las ventas (no cancelar pedidos hechos), dado que se interpreta como que el cliente es el que cancela el pedido al no presentarse y no el local.
\textbf{No hay actualizaciones simultáneas:} Hablar sobre esto.

\section*{Vistas}
\addcontentsline{toc}{section}{Vistas}

Cosas que dijo la catedra:

está sería la parte principal del TP. Acá irían los
diagramas y los escenarios representativos de uso. No necesariamente
tiene que ser una gran sección, sino que pueden partirla por
funcionalidades y a su vez por tipo de vista o de la forma que crean
más pertinente. Esta sección NO DEBE SER una simple seguidilla de
figuras sueltas. Debe estar acompañada de tantas explicaciones como
sean necesarias para que se aprecie un hilo conductor.

\section*{Modelo de objetivos}
\addcontentsline{toc}{subsection}{Modelo de obejtivos}
\subsection*{Objetivos}
\addcontentsline{toc}{subsubsection}{Obejtivos}
\noindent

Mantener TRAZABILIDAD! O sea, ponerle numeritos a las cosas.
Podriamos usar subsections en lugar de textbf, no?
\textbf{Mantener[Menú estandarizado]:} para mantener un menú estandarizado simplemente tenemos que aseguranos que sea el mismo en todos los locales, sean cual sean las actualizaciones que se han realizado (de variedades, precios, etc). Para ello únicamente tenemos que asegurar que se mantenga la comunicación entre los locales. \\
\textbf{Lograr[Se puede ingresar un menú desde una sucursal]:} para lograr un menú actualizable tenemos que asegurarnos de que se puedan realizar las actualizaciones pertinentes de éste. \\
\textbf{Lograr [Autorizar el ingreso de un menú nuevo desde una sucursal]:} pensamos varias formas de manejar esto: una de ellas consiste en que los locales pueden realizar actualizaciones en un momento determinado del día, bien puede ser esto a una hora prefijada de antemano o bien puede ser cuando cierra el local a la noche. Si el local no respeta estos horarios no le es permitido ingresar la actualización. Otro caso a considerar es cuando se cae la conexión del sistema, en caso de usar comunicación automatizada, en cuyo caso tampoco se permiten las actualizaciones, dado que sería una actualización local y el menú debe ser el mismo en todos los lugares. Por otra parte se debe verificar de alguna manera que quien intenta ingresar un nuevo menú esté autorizado. Puede utilizarse una contraseña o crear cuentas para los distintos usuarios del sistema. \\
\textbf{Lograr [Recargar stock al detectarse el nivel de stock]:} para esto planteamos dos alternativas. La primera es recargar el stock cuando se agote completamente, que es más eficiente en el manejo de stock porque involucra que se carga sólo las veces necesarias (recordar que cada carga tene su costo, por ejemplo el pago de transporte al proveedor). La segunda es recargar el stock cuando se detecte bajo stock, esto no es eficiente en términos de manejo de stock, pero ayudaría enormemente a mantener un menú rico en disponibilidad de productos, lo cual conllevaría a perder menos clientes debido a que sus pedidos no pudieron ser satisfechos.\\
\textbf{Lograr [Cancelar pedido a distancia]:} para cumplir este objetivo planteamos dos alternativas: por un lado el local es el que cancela el pedido al excederse un cierto \textit{time out}, mientras que por el otro el local llama periódicamente al cliente para confirmar que éste mantiene su interés. La primera se apoya en una presunción del dominio que indica que un cliente pierde el interés en realizar la compra luego de pasado cierto tiempo, en cuyo caso se cancela el pedido de forma automática. El segundo caso, en cambio, ofrece una mayor flexibilidad al cliente, que puede demorar más tiempo en llegar a la sucursal correspondiente. Pensamos que esta última opción de alguna forma mejora la satisfacción del cliente al permitirle mayor flexibilidad en el manejo de su tiempo.\\
\textbf{1.1.4.1.2.1 Lograr [Recargar stock al acabarse]:} mediante el agente controlador de stock se da aviso de la falta de stock. Los agentes pueden ser: una persona o un controlador de stock informatizado. El controlador de stock informatico (implementado en disitintas tipos de tecnologia) se comunica (mediante distipos tipos de comunicacion)con distintos proveedores.La persona llama al proveedor.\\
\textbf{1.1.4.1.2.2 Lograr [Recargar stock al detectarse stock bajo]:} idem anterior\\
\textbf{1.1.4.1.2 Lograr[Reponer stock]:}para cumplir este objetivo es necesaria una politica de reposicion.\\
\textbf{1.1.4.1.1 Mantener [monitoreo stock local]:} Se quiere lograr tener un control preciso de la disponibilidad de ingredietes. La estrategia a utilizar es la de descontar con cada orden los ingredientes utilizados dicha tarea se puede asignar a dos agentes diferentes. Por un lado esta tarea se puede asignar al personal del local para realizar el monitoreo de manera manual. Por otro lado, dicho objetivo puedo encontrarse a cargo de un controlador de stock informatizado en cada local que periodicamente se contrasta con un inventario total.\\
\textbf{1.1,4,1 Mantener [manejo de stock local]:} Se mantiene un monitoreo de la disponibilidad de ingredientes y se los repone cuando sea necesario.\\
\textbf{1.1.4.2 Mantener [monitoreo remoto]:} Se debera conocer el stock de las otras sucursales. \\
\textbf{1.1.4 mantener [manejo de stock] :} se debera conocerla disponibilidad de ingredientes tanto de la sucursal local como de las otras.y se debera mantener el stock en niveles aceptables.\\
\textbf{1.1.3 Lograr [reservar stock opara pedidos inmediatamente]:} Se debera reflejar la disminucion de la disponibilidad de ingredientes al momento de la produccion del pedido. \\
\textbf{1.1.2 Lograr [cancelar pedido a distancia]:} Se permitira cancelar el pedido a distancia en los casos que el cliente pierda el interes.\\
\textbf{1.1.2.2 Lograr [llamar periodicamente al cliente para que confirme su pedido]:}.Se contactara al cliente telefonicamente para uqe confirme que mantiene su interes por el pedido. este objetivo puede ser asignado al personal que debera ademas ocuparse de obtener la informacion de contacto.\\
\textbf{2.1.1 El Cliente pierde el interes despues del tiempo de espera:} Hipotesis de de dominio luego de un determinado tiempo se asume que el cliente no va a retirar el pedido.\\
\textbf{1.1.2.1.2 Lograr cancelar el pedido despues de un tiempo de espera:} Luego de expirar el tiempo de time out el controlador de stock envia una senal indicandole al local una actualizacion de stock.\\
\textbf{1.3.1.1.2 Sistema de sincornizacion de datos:}  mediante el sistema informatico de internet se mantiene la comunicacion automatica entre diferentes locales en su forma optima(en contraposicion al otro agente).\\

%Se debera permitir aceptar pedidos de una sucursal remota y a la vez poder realizar los pedidos que no sean satisfechos pro el local en cuestion a otra sucursal de la cadena que si disponga del stock necesario para cubrir el pedido del cliente.


\section*{Escenarios}

El dia 29/02/2004 a las 12:47:59 entran simultaneamente al local de pizza Hack (sucursal barrio mitre) Alan Faina, Sergio Mazza, Tomas Anchorena y 4taPersona en busqueda de una buena sapi. Alan se sienta en la mesa nro 1, tomas en la mesa nro 2, sergio mazza en la 3 y la 4ta en la mesa 4. 
\begin{itemize}\item
\item pedir en otra sucursal e ir
\item peidr en otra y no ir
\item pedir y que no haya
\item pedir y que haya y que el siguiente no tenga
\item que se quede sin stock y pedir stock.
\end{itemize}

\
\section*{Discusi\'on}
\addcontentsline{toc}{section}{Discusi\'on}

Cosas que dijo la catedra:

debe contener un análisis general de las distintas
alternativas y su impacto en los objetivos blandos. NO DEBE SER una
lista de lo que ya se puede desprender del diagrama. Debe ser un
análisis más cualitativo y "a vuelo de pájaro" que permita entender
este asunto en pocas palabras. Esta sección también es ideal para que
vuelquen las cosas que puedan haberles quedado "flojas" o no cerradas
del todo: por ejemplo los potenciales conflictos entre objetivos, los
aspectos que podrían hacer que todo el sistema no funcione como
desean, etc.


\section*{Conclusiones}
\addcontentsline{toc}{section}{Conclusiones}

Cosas que dijo la catedra:

mencionar brevemente de qué formas les resultó más
sencillo encarar el TP. Por ejemplo en qué orden realizaron los
diagramas, qué aspectos presentaron las mayores dificultades, etc.




\end{document}
