Aca quedan pendientes del MR.\\

TODO: Chequear si la foreing key en las debiles va subrayada o no (dado q en el manual no esta subrayada)
TODO: Hacer Restricciones de la relacion unaria \\
TODO: No se como poner la referencia de CAUSAS me marece lo de la interrelacion unaria.
\begin{itemize}

\item TELÉFONO\_J( \underline{idTel}, \underline{númeroJuzgado}, númeroTel) 
	\begin{itemize}
		\item CK = \{(idTel, númeroJuzgado)\}
		\item PK = \{(idTel, númeroJuzgado)\}
		\item FK = \{(númeroJuzgado)\}
		\item númeroJuzgado hace referencia a JUZGADO.númeroJuzgado.
		\item Restricciones:
			\begin{itemize}
			\item  TELÉFONO\_J.númeroJuzgado debe estar en JUZGADO.númeroJuzgado.
			\\
			
			\end{itemize}
	\end{itemize}	

	
\item JUZGADO( \underline{númeroJuzgado}, fechaCreación, dirección, \dashuline{idSala})
	\begin{itemize}
		\item CK = \{(númeroJuzgado)\}
		\item PK = \{(númeroJuzgado)\}
		\item FK = \{(idSala)\}
		\item Restricciones:
		\item idSala hace referencia a SALA.idSala.
			\begin{itemize}
			\item JUZGADO.númeroJuzgado puede no estar en TELÉFONO\_J.númeroJuzgado.
			\item JUZGADO.idSala no puede ser NULO.
			\item JUZGADO.númeroJuzgado debe estar en ACARGO.númeroJuzgado.
			\\
			\end{itemize}
	\end{itemize}


\item TELÉFONO\_S( \underline{idTel}, \underline{idSala}, númeroTel) 
	\begin{itemize}
		\item CK = \{(idTel, idSala)\}
		\item PK = \{(idTel, idSala)\}
		\item FK = \{(idSala)\}
		\item idSala hace referncia a SALA.idSala.
		\item Restricciones:
			\begin{itemize}
			\item TELÉFONO\_S.idSala debe estar en  SALA.idSala.
			\\
			\end{itemize}
	\end{itemize}


\item SALA( \underline{idSala}, nombreSala, \dashuline{idCámara},dirección)
	\begin{itemize}
		\item CK = \{(idSala)\}
		\item PK = \{(idSala)\}
		\item FK = \{(idCámara)\}
		\item idCámara hace referencia a CÁMARA.idCámara.
		\item Restricciones:
			\begin{itemize}
			\item SALA.idSala puede no estar en TELÉFONO\_S.idSala.
			\item SALA.idCámara no puede ser NULO.
			\\
			\end{itemize}
	\end{itemize}
	
	
\item TELÉFONO\_C( \underline{idTel}, \underline{idCámara}, númeroTel) 
	\begin{itemize}
		\item CK = \{(idTel, idCámara)\}
		\item PK = \{(idTel, idCámara)\}
		\item FK = \{(idCámara)\}
		\item idCámara hace referencia referencia a CÁMARA.idCámara.
		\item Restricciones:
			\begin{itemize}
			\item TELÉFONO\_C.idCámara debe estar en CÁMARA.idCámara.
			\\
			\end{itemize}
	\end{itemize}


\item CÁMARA( \underline{idCámara}, nombreCámara, dirección)
	\begin{itemize}
		\item CK = \{(idCámara)\}
		\item PK = \{(idCámara)\}
		\item FK = \{ \}
		\item Restricciones:
			\begin{itemize}
			\item CÁMARA.idCámara debe estar en TELÉFONO\_C.idTeléfono.
			\item CÁMARA.idCámara puede no estar en CONCURSO.idCámara.
			\\			
			\end{itemize}
	\end{itemize}
		
	
\item CONCURSO( \underline{idConcurso},\dashuline{idCámara})
	\begin{itemize}
		\item CK = \{(idConcurso)\}
		\item PK = \{(idConcurso)\}
		\item FK = \{(idCámara)\}
		\item idCámara hace referencia a CÁMARA.idCámara.
		\item Restricciones:
			\begin{itemize}
			\item CONCURSO.idCámara no puede ser NULO.
			\item CONCURSO.idConcurso puede no estar en INSCRIPCIÓN.idConcurso.
			\\			
			\end{itemize}		
	\end{itemize}
	
	
\item INSCRIPCIÓN( \underline{idInscripción}, ordenMérito, nombreUniversidad, promedio, fechaTitulo, \dashuline{idConcurso}, \dashuline{cuil})
	\begin{itemize}
		\item CK = \{(idInscripción)\}
		\item PK = \{(idInscripción)\}
		\item FK = \{(idConcurso),(cuil)\}
		\item idConcurso hace referencia a CONCURSO.idConcurso.
		\item cuil hace referencia a ABOGADO.cuil.
		\item Restricciones:
			\begin{itemize}
			\item INSCRIPCIÓN.idConcurso no puede ser NULO.
			\item INSCRIPCIÓN.cuil no puede ser NULO.
			\\
			\end{itemize}
	\end{itemize}
	
	
\item ABOGADO( \underline{cuil}, númerodeLegajo, telParticular, apellido, nombre)
	\begin{itemize}
		\item CK = \{(cuil)\}
		\item PK = \{(cuil)\}
		\item FK = \{\}
		\item Restricciones:
			\begin{itemize}
			\item ABOGADO.cuil puede no estar en INSCRIPCIÓN.cuil.
			\item ABOGADO.cuil puede no estar en ACARGO.cuil.
			\item ABOGADO.cuil puede no estar en SECRETARIO.cuil.
			\\
			\end{itemize}
	\end{itemize}

	
\item NORMA( \underline{númeroNorma}, fechaPublicación, tipo)
	\begin{itemize}
		\item CK = \{(númeroNorma)\}
		\item PK = \{(númeroNorma)\}
		\item FK = \{\}
		\item Restricciones:
			\begin{itemize}
			\item NORMA.númeroNorma puede no estar en ACARGO.númeroNorma.
			\item NORMA.númeroNorma puede no estar en SECRETARIO.númeroNorma.
			\\
			\end{itemize}
	\end{itemize}

	
\item ACARGO(  \underline{númeroNorma}, \underline{númeroJuzgado}, \dashuline{cuil})
	\begin{itemize}
		\item CK = \{(númeroNorma,númeroJuzgado)\}
		\item PK = \{(númeroNorma,númeroJuzgado)\}
		\item FK = \{(cuil)\}
		\item cuil hace referencia a ABOGADO.cuil.
		\item Restricciones:
			\begin{itemize}
			\item ACARGO.númeroJuzgado debe estar en JUZGADO.númeroJuzgado.
			\item ACARGO.númeroNorma debe estar en NORMA.númeroNorma.
			\item ACARGO.cuil debe estar en ABOGADO.cuil.
			\\
			\end{itemize}
	\end{itemize}
	
	
\item SECRETARIO( \underline{númeroNorma}, \underline{idSecretaría}, \dashuline{cuil})
	\begin{itemize}
		\item CK = \{(númeroNorma,idSecretaría)\}
		\item PK = \{(númeroNorma,idSecretaría)\}
		\item FK = \{(cuil)\}
		\item cuil hace referencia a ABOGADO.cuil.
		\item Restricciones:
			\begin{itemize}
			\item SECRETARIO.númeroNorma debe estar en NORMA.númeroNorma.
			\item SECRETARIO.cuil debe estar en ABOGADO.cuil.
			\item SECRETARIO.idSecretaría debe estar en SECRETARÍA.
			\\
			\end{itemize}
	\end{itemize}


\item SECRETARÍA( \underline{idSecretaría}, \dashuline{númeroJuzgado})
	\begin{itemize}
		\item CK = \{(idSecretaría)\}
		\item PK = \{(idSecretaría)\}
		\item FK = \{(númeroJuzgado)\}
		\item númeroJuzgado hace referencia JUZGADO.númeroJuazgado.
		\item Restricciones:
			\begin{itemize}
			\item SECRETARÍA.númeroJuzgado no puede ser NULO.
			\item SECRETARÍA.idSecretaría puede no estar en CAUSAS.idSecretaría.
			\item SECRETARÍA.idSecretaría debe estar en SECRETARIO.idSecretaria.
			\\
			\end{itemize}
	\end{itemize}

	
\item CAUSAS( \underline{númeroCausa}, fechaApertura, descripción, \dashuline{númeroCausaOriginal}, \dashuline{idSecretaría})
	\begin{itemize}
		\item CK = \{(númeroCausa)\}
		\item PK = \{(númeroCausa)\}
		\item FK = \{(númeroCausaOriginal),(idSecretaría)\}
		\item 
		\item Restricciones:
			\begin{itemize}
			\item CAUSAS.idSecretaría no puede ser NULO.
			\item CAUSAS.númeroCausa puede no estar en MOVIMIENTOS.númeroCausa.
			\\
			\end{itemize}
			TODO: Las Restricciones de las unarias requieren definir roles creo, lo cuelgo.
	\end{itemize}

	
\item MOVIMIENTO( \underline{idMovimiento}, descripción, fecha, \dashuline{númeroCausa}, tipo)
	\begin{itemize}
		\item CK = \{(idMovimiento)\}
		\item PK = \{(idMovimiento)\}
		\item FK = \{(númeroCausa)\}
		\item númeroCausa hace referencia a CAUSA.númeroCausa.
		\item Restricciones:
			\begin{itemize}
			\item MOVIMIENTO.idMovimiento puede no estar en DECLARACIONES.idMovimiento.
			\\
			\end{itemize}
	\end{itemize} 
	
\item DECLARACIONES(\underline{idMovimiento}, dni, motivo, apellido, nombre)
	\begin{itemize}
		\item CK = \{ \}
		\item PK = \{ \}
		\item FK = \{ \}
		\item Restricciones:
			\begin{itemize}
			\item DECLARACIONES.idMovimiento debe estar en MOVIMIENTO.idMovimiento.
			\\
			\end{itemize}
	\end{itemize}
	
	
\end{itemize}		