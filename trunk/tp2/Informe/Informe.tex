\documentclass[a4paper, 12pt]{article}
\usepackage{graphicx}
\usepackage{verbatim}
% \usepackage{lstlisting}

\begin{document}
\begin{center}
\section*{Aclaraciones Generales}
\end{center}

\newpage
\begin{center}
\section*{Ejercicio 1: Matching M\'aximo}
\end{center}
 
\section*{Introducci\'on}
En este ejercicio se ped\'ia encontrar el matching m\'aximo dentro de un grafo. Se define como matching a un subconjunto de aristas que comparten v\'ertices. Si bien este ejercicio podr\'ia ser de gran complejidad, el mismo se encuentra en una presentaci\'on mas accesible dado que no hay que aplicar el algoritmo sobre cualquier tipo de grafos, sino que solamente tiene que ser aplicable a grafos de 3 o m\'as nodos que sean ciclos simples, es decir, grafos que en su isomorfismo planar sean simplemente un dibujo de un pol\'igono. 

/% aca podria ir algun dibujo de como serian los grafos a ver y como seria algun grafo que no hay que analizar.

\section*{Algoritmo}
A continuaci\'on se muestra el pseudoc\'odigo del algoritmo propuesto como resoluci\'on del problema.


% \begin{lstlisting}
Aca va el pseudocodigo.
% \end{lstlisting}


Este algoritmo consiste en transformar el grafo en un vector, dado que al poder ser representado como un pol\'igono, se puede ver que, quitando una sola arista, el grafo se convierte en una sucesi\'on de aristas con peso, lo cual se puede representar con un vector de n\'umeros. La idea del algoritmo es poder encontrar el m\'aximo matching posible, una primera aproximaci\'on a la soluci\'on final, podr\'ia ser la siguiente:

\begin{enumerate}
\item Se toma una arista cualquiera para comenzar. Luego, el matching buscado tiene dos opciones: contener esa arista, o no contenerla. Entonces el resultado ser\'a el m\'aximo entre el matching m\'aximo del grafo sin esa arista(no se utiliza la primer arista tomada) y el matching m\'aximo del grafo sin esa arista, ni ninguno de sus dos aristas vecinas, m\'as el valor de la arista elegida en primer t\'ermino(si se utiliza la primer arista tomada). De esta manera, el problema sobre un grafo, se convierte en 2 problemas similares pero sobre vectores.
\item En este momento, se necesita sacar el matching m\'aximo, pero sobre vectores, para hacer esto se piensa de manera parecida al punto anterior comenzando con el \'ultimo elemento: o bien el matching m\'aximo lo contiene, o bien no lo contiene. Luego, el matching buscado para el vector, sera el m\'aximo entre el matching m\'aximo del vector sin el \'ultimo elemento(caso en el que no se usa el \'ultimo elemento) y el matching m\'aximo del vector sin los \'ultimos dos elementos m\'as el valor del \'ultimo elemento(caso en el que si se utiliza el \'ultimo).

	$matchingMaximo(v_1,v_2,..,v_n) = maximo(matchingMaximo(v_1,v_2,..,v_{n-1}), matchingMaximo(v_1,v_2,..,v_{n-2}) + v_n)$
\item De esta manera, se puede ver que el problema se torna recursivo, siendo solucionado mediante la t\'ecnica de dividir y conquistar, teniendo como caso base los vectores de dos o un elemento resolubles trivialmente.
\end{enumerate}

Si bien el algoritmo propuesto retorna el valor esperado, la complejidad del mismo no es \'optima. Al hacer los llamados recursivos sucede que hay varios matching m\'aximos que se realizan sobre los mismos vectores, generando as\'i m\'as c\'alculos de los necesarios. Fue por esto, que el siguiente paso fue modificar el algoritmo para que no calcule las cosas de modo \emph{top down}, sino que fuese un algoritmo \emph{bottom up}, para as\'i evitar los c\'alculos repetidos, utilizando as\'i la mayor ventaja de la programaci\'on din\'amica, t\'ecnica que define al algoritmo final.

De esta manera, se pens\'o como teniendo instancias m\'as peque~{n}as del problema, se puede conocer la soluci\'on de una instancia mayor. Se utiliz\'o que, dadas las soluciones \'optimas para vectores de n-2 y n-1 elementos, la soluci\'on para el vector de n elementos es el m\'aximo entre la solucion de n-1 elementos, y la soluci\'on de n-2 elementos m\'as el elemento n-\'esimo. Es as\' que el c\'alculo se torna \emph{bottom up}, calculando los m\'aximos de los subvectores, una sola vez.

Aca podria ir un ejemplo hecho paso por paso o algo asi.

\section*{Demostraci\'on de correctitud}
\section*{Complejidad}
\section*{An\'alisis de resultados}

Para analizar este algoritmo, se bas\'o el enfoque en dos aspectos diferentes. Por un lado, se encuentran los an\'alisis sobre la correctitud de la soluci\'on propuesta y, por otro lado, se encuentra el an\'alisis sobre el tiempo de ejecuci\'on para diferentes archivos de entrada, para poder realizar as\'i, una correlaci\'on entre el tiempo de ejecuci\'on y la complejidad te\'orica calculada anteriormente.

\subsection*{Casos de correctitud}
Con el fin de realizar una comprobaci\'on emp\'irica de la soluci\'on propuesta se genero un archivo de entrada con diez casos de prueba.
\begin{itemize}
\item Los primeros nueve casos de prueba, estan conformados por los entregados por la c\'atedra, teniendo de esta manera las soluciones reales para contrastar con las arrojadas por el algoritmo.
\item El \'ultimo caso de prueba esta conformado por un grafo de 30 aristas, con pesos de 1 a 30, ordenados consecutivamente. En este caso, se puede ver que el matching m\'aximo esta dado por tomar las aristas con valor par.
\end{itemize}

El an\'alisis en este tipo de casos se baso solamente en la correctitud de los mismos y no en el tiempo de ejecuci\'on debido a que son grafos de tama\~{n}os muy peque\~{n}os y el tiempo de ejecucion no sobrepasa los 2 microsegundos en ninguno de los casos.

Cabe destacar, que para todos los casos propuestos, los resultados fueron satisfactorios al ser contrastados con la soluciones previamente obtenidas

\subsection*{Casos de Stress}.

\section*{Conclusiones}


\begin{center}
\item \section*{Ejercicio 2: Se inunda la isla}
\end{center}

\section*{Introducci\'on}
En este ejercicio se pedia encontrar el area que no se inunde en una isla plana luego de ciertas condiciones.

En primer lugar, se tiene una isla que posee la misma altura en todos sus puntos, es por esto que si la marea sube la isla se inunda por completo. La soluci\'on para que las partes importantes de la isla no se inunde cuando sube la marea, es poner una serie de vallas rectangulares que no dejen pasar el agua hasta cierta altura. La idea ser\'ia entonces, colocando adecuadamente estas vallas, encerrar partes de la isla para que el agua no pueda entrar.

Aca se podria poner un dibujo de ejemplo de la isla con ciertas vallas o algo asi.

El problema consiste en, dado un conjunto de vallas y el nivel de la marea. Calcular cual es el area de la isla que no va a ser inundada.

\section*{Algoritmo}
A continuaci\'on se presenta el pseudoc\'odigo y la explicaci\'on de la soluci\'on propuesta. 


Dado que, por restricci\'on del problema, las vallas no pueden estar en cualquier lado, sino que su coordenada (x,y) del punto inferior izquierdo esta formada por x e y enteros. Asimismo, como la longitud de una valla tambien es entera, la coordenada del v\'ertice restante tambien ser\'a entera. De esta manera, podemos ver que la isla se puede pensar como una grilla de cuadrados de 1x1. Luego, esta grilla fue pensada como un grafo, donde cada cuadrado de la grilla es un nodo y las aristas estan dadas por la relaci\'on entre un cuadrado y sus 4 posibles vecinos. Para armar esta grilla, se calcularon los m\'inimos y m\'aximos valores de las vallas en x y en y, ya que por fuera de estos los cuadrados que pertenezcan a la isla se inundar\'an de todos modos (ya que no hay vallas que los cubran).

Al comenzar el algoritmo, todos los cuadrados estan relacionados con sus vecinos con aristas de peso 0. Esto quiere decir que si un cuadrado se inunda, los cuadrados que esten relacionados con ese por una arista de peso 0 tambien se va a inundar. 

Luego, se recorren todas las vallas dadas por el problema y se setean las nuevas relaciones entre los cuadrados, es decir, si existe una valla de altura 4 entre el nodo $v_1$ y el nodo $v_2$, lo que se hace es ponerle un peso de 4 a la arista que los relaciona. Indicando as\'i que solamente si la marea es mayor a 4, el agua pasara de ese $v_1$ a $v_2$ directamente.

Por \'ultimo, lo que se hace es crear una circunvalaci\'on de nodos que se inundan al rededor del grafo real y realizar BFS desde uno de estos (que seguro se inunda) y contar cuantos nodos tiene la componente conexa que el BFS recorre por completo. Obviamente, este BFS toma que un nodo es vecino de otro si la arista que los une tiene peso menor a la marea, sino podremos decir que la valla es efectiva entre esos dos nodos y no hay inundaci\'on de uno hacia el otro. Una vez obtenida la cantidad total de nodos dados por el recorrido en anchura, el \'ultimo paso es realizar la sustracci\'on entre los nodos totales de la grilla, y los nodos inundados; obteniendo as\'i, la cantidad total de nodos que no fueron inundados gracias a la protecci\'on de las vallas. Por \'ultimo, como cada nodo representa un cuadrado de \'area 1, la cantidad de nodos no alcanzados por el BFS es igual al \'area no inundada.

\begin{verbatim}
{
	vector vallas;

	leer(vallas)

	maxmin(vallas)

	matriz nodo[alto][ancho];

	seteamosMatriz(matriz)

	int inundadas  <- bfsContador(matriz)


	return (ancho*alto - inundadas)
}
\end{verbatim}
\begin{itemize}

\item leer(vallas): Carga en el vector vallas, todas las vallas del input.

\item maxmin(vallas): Setea ciertas variables con el tama\~no m\'aximo que tendrá la isla delimitada por las vallas mas externas.

\item seteamosMatriz(matriz): Usando las variables seteadas por maxmin crea un grafo (representado con una matriz) donde establece 4 relaciones por nodo (con sus vecinos) donde el peso del a arista es la altura de la valla que debe atravesar.

\item bfsContador(matriz): Recorre el grafo usando bfs, por cada nodo que visita suma uno a una variable que al final devuelve. Recorre todos los nodos que se inundan dado q son los que se relacionan (si la marea supera el peso de las aristas). De este modo al finalizar obtenemos cuantos nodos visitamos coincidiendo con cuantos nodos se inundan.

\end{itemize}


\section*{Complejidad}
Para analizar la complejidad se utiliz\'o el modelo uniforme. En este modelo el an\'alisis no esta centrado en el tama\~{n}o de los operandos, por lo que el tiempo de ejecuci\'on de cada operaci\'on se considera constante.

Si tomamos como tama\~no de entrada la cantidad de vallas (CV, en adelante), y como sabemos que estas no se solapan en mas de un punto, podemos acotar (por arriba) a la cantidad de vallas, por la cantidad de nodos en el grafo. Por consiguiente vamos a analizar el algoritmo usado en funcion a la cantidad de nodos. TODO JUSTIFICAR BIEN ESTO
La funci\'on leer, leer del archivo de entrada toda una instancia del problema ciclando CV para poder relevar la informacion de las vallas, por lo cual esta operaci\'on la realizamos en O( CV ) 
La funci\'on maxmin, busca m\'aximos y m\'inimos en un vector de vallas linealmente, por lo cual es de orden O ( CV ).
...
habia q seguir

\section*{An\'alisis de resultados}
\section*{Conclusiones}

\begin{center}
\item \section*{Ejercicio 3: Bernardo Armando Pandillas}
\end{center}

\section*{Introducci\'on}


En este problema lo que se quiere es, teniendo un conjunto de personas y sabiendo si se conocen entre s\'i, armar dos grupos de tres personas cada uno, un grupo en el cual las 3 personas se conozcan mutuamente, y otro grupo en el cual ninguno est\'e relacionado con otro.


Si se pudieran armar varios grupos que cumplan con la restricci\'on planteada, el que debemos encontrar es el menor lexicogr\'aficamente. Es decir, si representamos cada grupo con una terna ordenada ascendentemente, la terna ser\'a menor que cualquier otra terna que represente a un grupo v\'alido.

Consideramos que una terna es menor que otra si la primera componente de la primera terna es menor, o si es igual y la segunda de la misma es menor, o si tanto la primera como la segunda son iguales y la tercera es menor, en cada caso con respecto a la misma componente de la otra terna.
\begin{displaymath}
\left(a, b, c \right) < \left( d, e, f\right) \Leftrightarrow 
\left( a < d \right) \vee \left( a = d \wedge b < e \right) \vee \left( a = d \wedge b = e \wedge c < f \right)
\end{displaymath}


\section*{Algoritmo}


Para resolver el problema se opta por representarlo mediante un grafo.

Cada v\'ertice del grafo representa a una persona diferente y dos v\'ertices est\'an unidos si y s\'olo si las personas se conocen.


Se puede ver que existe un clique de tres nodos si y s\'olo si las personas representadas por los nodos que est\'en en dicho clique se conocen todas entre s\'i. Ver Ap\'endice \ref{dem_clique}.


Para encontrar el grupo de tres personas que no se conozcan entre s\'i resolvemos el mismo problema pero en el grafo complemento. Ver Ap\'endice \ref{dem_clique}.


Sabiendo que el modelo que elegimos es correcto resolvemos el problema de encontrar un clique de tres nodos en el grafo.


Se confecciona la matriz de adyacencias, llam\'emosla $A$.


El i-\'esimo nodo representa a la i-\'esima persona.


Sea $C = A^3$, $C$ tiene en la i-\'esima posici\'on de su diagonal la cantidad de caminos de longitud 3 desde el nodo i hasta el nodo i. \footnote{Te\'orica Algoritmos III}   

 
Buscamos en $C$ el m\'inimo i tal que $C_{i,i} \neq 0$, llam\'emoslo j. Si existe, entonces, el menor clique de 3 nodos contendr\'a a j. Si no fuera as\'i, entonces existe k < j tal que $C_{k,k} \neq 0$, lo cual es absurdo, ya que j es el m\'inimo i tal que $C_{i,i} \neq 0$. Si no existe, termina el algoritmo indic\'andolo de la manera correspondiente.


De todas las combinaciones posibles para los otros dos nodos, elegimos la menor tal que los tres nodos forman una clique. En nuestra implementaci\'on optimizamos esta ultima selecci\'on, evitando generar combinaciones que sabemos anticipadamente que no corresponden a un clique. De esta manera obtuvimos el grupo de tres personas que se conocen entre s\'i.


Por \'ultimo obtenemos el grafo complemento de $G$, y realizamos las mismas operaciones para obtener el grupo de tres personas que no se conocen entre s\'i.



\section*{Complejidad}


Obtenemos $C = A^3$ utilizando el algoritmo de Strassen para realizar el producto de matrices. La cantidad de operaciones que realiza este algoritmo est\'a en $O\left(m^{log_2\left(7\right)}\right)$, para una matriz de m filas y m columnas.\footnote{Ver Brassard, p\'ag. 273.} Si el grafo tiene n nodos, es necesario generar una matriz de m x m, con $m \leq 2n$, ya que es necesario que n sea una potencia de dos. De esta forma, la cantidad de operaciones para realizar el producto de matrices est\'a en $O\left(7 \times \left(n ^{log_2 \left( 7\right)} \right) \right) = O\left(n^{log_2\left(7\right)}\right)$.



Se puede encontrar en $C$ el m\'inimo i tal que $C_{i,i} \neq 0$ en $O\left(n\right)$ operaciones, ya que para cada posici\'on de la diagonal se realiza una cantidad constante de operaciones para verificar si en dicha posici\'on hay un cero. Como m\'aximo se recorre toda la diagonal, que tiene n posiciones.


Generando todas las combinaciones de dos nodos la cantidad de operaciones para obtener la menor est\'a en $O\left(n^2\right)$, ya que se requiere una cantidad constante de operaciones para generar cada combinaci\'on, verificar si es un clique y compararla con la menor hasta el momento.


Por \'ultimo se vuelve a calcular lo mismo para el grafo complemento. Para lo cual se invierte la matriz de adyacencias en $O \left( n^2 \right)$.


Como se puede ver, la complejidad computacional del algoritmo en el modelo uniforme est\'a dada por las operaciones correspondientes a la multiplicaci\'on de matrices, en $O \left( n^{log_2\left(7\right)} \right) $, ya que el resto del algoritmo tiene un orden menor. Como $log_2\left(7\right) < 2.9 $, el algoritmo realiza una cantidad de operaciones de un orden estrictamente menor a $O\left( n^3\right) $.




\section*{An\'alisis de resultados}
\section*{Conclusiones}

\section*{Ap\'endices}
\section{Demostraci\'on de correctitud del modelo}\label{dem_clique}

Para resolver el ejercicio 3 se opta por representarlo mediante un grafo.

Cada v\'ertice del grafo representa a una persona diferente y dos v\'ertices est\'an unidos si y s\'olo si las personas se conocen.


A continuaci\'on se demuestra que existe un clique de tres nodos en el grafo que modela el problema si y s\'olo si las personas representadas por los nodos que est\'en en dicho clique se conocen todas entre s\'i. Luego se demuestra que existe un clique de tres nodos en el complemento del grafo planteado si y s\'olo si las personas representadas por los nodos que est\'en en dicho clique no se conocen entre s\'i.

Sea $G = \left( V, E \right)$ el grafo con el que modelamos el problema. Sea $ C = \left( \left\lbrace v_a, v_b,v_c \right\rbrace, E_c \right) $ un clique de tres nodos donde C es subgrafo de G y donde $v_a$ representa a la persona $a$, $v_b$ a la persona $b$ y $v_c$ a la persona $c$. Como $\left( v_a, v_b \right) \in E $, $a$ conoce a $b$. Tambi\'en $\left(v_b,v_c\right) \in E$, por lo que $b$ conoce a $c$. Por \'ultimo como $\left(v_c,v_a\right) \in E$, $c$ conoce a $a$. Por lo tanto las personas representadas por los nodos que est\'an en un clique de tres nodos se conocen entre s\'i de a pares.


A su vez, si tres personas, llam\'emoslas $a$, $b$ y $c$, se conocen entre s\'i de a pares, existe un clique de tres nodos que contiene a los nodos que las representan. 


Sea $ G = \left( V, E \right) $ el grafo con el que modelamos el problema. Sean $v_a, v_b, v_c$ los nodos que las representan, como $a$ conoce a $b$, $\left( v_a,v_b \right) \in E $. Como $b$ conoce a $c$, $ \left(v_b, v_c \right) \in E$. Como tambi\'en $c$ conoce a $a$, $ \left( v_c, v_a \right) \in E$, por lo tanto $\left( \left\lbrace v_a, v_b, v_c \right\rbrace, \left\lbrace  \left(v_a,v_b\right) , \left( v_b,v_c\right), \left( v_c, v_a\right) \right\rbrace\right)$ es un clique de 3 nodos subgrafo de G.


Resta demostrar que existe un clique de tres nodos en el complemento del grafo planteado si y s\'olo si las personas representadas por los nodos que est\'en en dicho clique no se conocen entre s\'i.


Llamemos $G' = \left( V', E')$ al complemento de $G$.  Sea $ C = \left( \left\lbrace v_a, v_b,v_c \right\rbrace, E' \right) $ un clique de tres nodos donde C es subgrafo de G y donde $v_a$ representa a la persona $a$, $v_b$ a la persona $b$ y $v_c$ a la persona $c$. Como $\left( v_a,v_b\right) \in E'$, $a$ y $b$ no se conocen. Tambi\'en $\left( v_b,v_c\right) \in E'$, por lo que $b$ y $c$ no se conocen. Por \'ultimo $\left( v_c,v_a\right) \in E'$, y por lo tanto $c$ y $a$ no se conocen. Luego $a$, $b$, y $c$ forman un grupo de tres personas que no se conocen.



Rec\'iprocamente, si sabemos que $a$, $b$ y $c$ forman un grupo de tres personas que no se conocen entre s\'i, y sean $v_a, v_b, v_c$ los nodos que las representan, entonces como $a$ y $b$ no se conocen, $\left(v_a,v_b\right) \in E'$. Tampoco $b$ y $c$ se conocen, por lo que $\left(v_b,v_c\right) \in E'$. Por \'ultimo $c$ y $a$ no se conocen, por lo cual $\left(v_c,v_a\right) \in E'$. Luego $\left( \left\lbrace v_a, v_b, v_c \right\rbrace, \left\lbrace  \left(v_a,v_b\right) , \left( v_b,v_c\right), \left( v_c, v_a\right) \right\rbrace\right)$ es un clique de tres, subgrafo de $G'$.

\end{document}
