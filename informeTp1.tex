\documentclass[a4paper, 12pt] {article}
\usepackage{a4wide}
\usepackage{graphicx}
\usepackage{verbatim}

\begin{document}

\begin{center}
\section*{Aclaraciones generales} 
\end{center}

\newpage

\begin{center}
\section*{Ejercicio 1: Intervalos}
\end{center}

\bigskip
\section*{Introduci\'on}

\section*{Algoritmo}

\begin{verbatim}
\end{verbatim}

\section*{Complejidad}

\subsection*{Modelo Uniforme}

\subsection*{Modelo Logaritmico}

\section*{An\'alisis de resultados}

\section*{Testing}
Se realizaron dos tipos de pruebas: En primer lugar, se analizaron los casos borde y luego lo de stress respecto al tama\~{n}o de la entrada en tiempo de ejecuci\'on.
Como caso borde se tomaron los casos que probablemente el algoritmo no estuviese abarcando o pudiese estar ignorando.
En este ejercicio los casos borde estudiados fueron los siguientes:
\begin{itemize}
\item[Caso 1:] v1()
\item[Caso n:] vn()
 ×
\end{itemize}

Mientras que los de stress fueron:
Caso 1:
Caso n:
Los resultados obtenidos son :

\begin{comment}
\begin{center}

\begin{center}

\end{center}
\end{center}
\end{comment}

\section*{Conclusiones}

\newpage

\begin{center}
\section*{Ejercicio 2: N\'umeros Amigos}
\end{center}

\bigskip
\section*{Introduci\'on}

\section*{Algoritmo}

\begin{verbatim}
\end{verbatim}

\section*{Complejidad}

\subsection*{Modelo Uniforme}

\subsection*{Modelo Logaritmico}

\section*{An\'alisis de resultados}

\section*{Testing}

\begin{comment}
\begin{center}
\includegraphicx[width=0.7\textwidth]{Plots/Ej1-Complejidad.png}
\begin{center}
FiguraX
\end{center}
\end{comment}

\section*{Conclusiones}
\newpage

\begin{center}
 \section*{Ejercicio 3: Tableros}
\end{center}

\bigskip
\section*{Introduci\'on}

\section*{Algoritmo}

\begin{verbatim}
\end{verbatim}

\section*{Complejidad}

\subsection*{Modelo Uniforme}

\subsection*{Modelo Logaritmico}

\section*{An\'alisis de resultados}

\section*{Testing}

\begin{comment}
\begin{center}
\includegraphicx[width=0.7\textwidth]{Plots/Ej1-Complejidad.png}
\begin{center}
FiguraX
\end{center}
\end{center}
\end{comment}
\section*{Conclusiones}

\end{document}