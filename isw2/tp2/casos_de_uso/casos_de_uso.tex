%~ 
%~ \documentclass{article}
%~ 
%~ \usepackage[spanish, activeacute]{babel}
%~ \usepackage[utf8x]{inputenc}
%~ \usepackage{booktabs}
%~ 
%~ \begin{document}
%~ 

%~  Que un usuario se loguee

Ingresando al sistema.



\begin{tabular}{ll}
Actor & Votante \\
\hline
Pre condición & Ninguna \\
\hline
Pos condición & El usuario se encuentra autenticado. \\
\hline
\end{tabular}




\begin{tabular}{c p{4cm}|p{4cm}}
 & Curso normal & Curso alternativo \\
 1. & El votante indica su nombre de usuario y contraseña &  \\
 2. & Si los datos son correctos el usuario recibe una confirmación de que su nombre de usuario y contraseña son correctos & Si los datos son incorrectos el usuario recibe un mensaje de error indicándo que la combinación de nombre de usuario y contraseña es incorrecta \\
 3. & Se muestra la interfaz de voto electrónico & Ir al fin del caso de uso \\
4. & Fin del caso de uso& \\ 
\end{tabular}

\bigskip

%~  Que un usuario vote
Votando:


\begin{tabular}{ll}
Actor & Votante \\
\hline
Pre condición & El usuario se encuentra autenticado \\
\hline
Pos condición & Se registra correctamente el voto del usuario \\
\hline
\end{tabular}




\begin{tabular}{c p{4cm}|p{4cm}}
 & Curso normal & Curso alternativo \\
 1. & Se muestra ante el usuario la lista de candidatos &   \\
 2. & El usuario selecciona el candidato de su preferencia &   \\
 3. & Se pide confirmación al usuario mostrando claramente qué candidato ha elegido &   \\
 4. & El votante confirma su selección & Si el usuario no confirma, se descarta la selección del usuario  \\
 5. & Se le informa al usuario que se ha enviado a su e\-mail el certificado del voto & Ir al fin del caso de uso \\
 6. & Fin del caso de uso & \\
\end{tabular}


\bigskip

%~  Que una agrupacion vea los resultados parciales
Fiscalizando resultados parciales:

En este caso de uso se accede desde la terminal de una agrupación política a una interfaz gráfica que permite ver los resultados disponibles en tiempo real.










%~ \end{document}


