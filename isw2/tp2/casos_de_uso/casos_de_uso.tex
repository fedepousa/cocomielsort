\section{Casos de Uso}

A continuaci\'on se presentan los principales casos de uso del sistema.
Se presentan las principales interacciones que hay desde el exterior(votantes, rectorado, partidos pol\'iticos) con el sistema.

\begin{itemize}
\item Un usuario quieren ingresar al sistema
\begin{center}
\begin{tabular}{ll}
Actor & Votante \\
\hline
Pre condición & Ninguna \\
\hline
Pos condición & El usuario se encuentra autenticado. \\
\hline
\end{tabular}
\medskip
\begin{tabular}{c p{4cm}|p{4cm}}
 & Curso normal & Curso alternativo \\
 1. & El votante indica su nombre de usuario y contraseña &  \\
 2. & Si los datos son correctos el usuario recibe una confirmación de que su nombre de usuario y contraseña son correctos & Si los datos son incorrectos el usuario recibe un mensaje de error indicándo que la combinación de nombre de usuario y contraseña es incorrecta \\
 3. & Se muestra la interfaz de voto electrónico & Ir al fin del caso de uso \\
4. & Fin del caso de uso& \\ 
\end{tabular}
\end{center}

\bigskip
\item Un usuario quiere emitir un voto v\'alido.
\begin{center}
\begin{tabular}{ll}
Actor & Votante \\
\hline
Pre condición & El usuario se encuentra autenticado \\
\hline
Pos condición & Se registra correctamente el voto del usuario \\
\hline
\end{tabular}
\medskip
\begin{tabular}{c p{4cm}|p{4cm}}
 & Curso normal & Curso alternativo \\
 1. & Se muestra ante el usuario la lista de candidatos &   \\
 2. & El usuario selecciona el candidato de su preferencia &   \\
 3. & Se pide confirmación al usuario mostrando claramente qué candidato ha elegido &   \\
 4. & El votante confirma su selección & Si el usuario no confirma, se descarta la selección del usuario  \\
 5. & Se le informa al usuario que se ha enviado a su e\-mail el certificado del voto & Ir al fin del caso de uso \\
 6. & Fin del caso de uso & \\
\end{tabular}
\end{center}

\bigskip
\item Fiscalizaci\'on de los resultados parciales.
\begin{center}
\begin{tabular}{ll}
Actor & El usuario del fiscalizador \\
\hline
Pre condición & El usuario se encuentra autenticado \\
\hline
Pos condición & Se entregan resultados parciales \\
\hline
\end{tabular}
\medskip
\begin{tabular}{c p{4cm}|p{4cm}}
 & Curso normal & Curso alternativo \\
 1. & Se muestran opciones de resultados parciales por facultad con votaciones abiertas &   \\
 2. & El usuario selecciona el resultado parcial a observar &   \\
 3. & Fin del caso de uso & \\
\end{tabular}
\end{center}


\bigskip
\item El rectorado quiere enviar nuevos reglamentos a las facultades.
\begin{center}
\begin{tabular}{ll}
Actor & El usuario del rectorado \\
\hline
Pre condici\'on & El usuario se encuentra autenticado \\
\hline
Pos condici\'on & Las facultades reciben el nuevo reglamento \\
\hline
\end{tabular}
\medskip
% \begin{tabular}{c p{4cm}|p{4cm}}
%  & Curso normal & Curso alternativo \\
%  1. & Se muestran opciones de resultados parciales por facultad con votaciones abiertas &   \\
%  2. & El usuario selecciona el resultado parcial a observar &   \\
%  3. & Fin del caso de uso & \\
% \end{tabular}
\end{center}

\bigskip
\item Cambio de idioma en la interfaz de usuario.
\begin{center}
\begin{tabular}{ll}
Actor & Votante \\
\hline
Pre condición & El usuario se encuentra autenticado \\
\hline
Pos condición & La interfaz se presenta en el usuario elegido \\
\hline
\end{tabular}
\medskip
\begin{tabular}{c p{4cm}|p{4cm}}
 & Curso normal & Curso alternativo \\
 1. & El usuario selecciona la funcionalidad de modificar el idioma del sistema &   \\
 2. & El sistema despliega los diferentes idiomas disponibles &   \\
 3. & El usuario elije entre uno de las opciones disponibles & \\
 4. & El sistema modifica la interfaz del usuario con el idioma elegido & \\
 5. & Fin de caso de uso & \\
\end{tabular}
\end{center}



\bigskip
\item Conexi\'on desde diversas plataformas
\begin{center}
\begin{tabular}{ll}
Actor & Votante \\
\hline
Pre condición &  \\
\hline
Pos condición & El usuario ingresa al sistema \\
\hline
\end{tabular}
\medskip
\begin{tabular}{c p{4cm}|p{4cm}}
 & Curso normal & Curso alternativo \\
 1. & Un usuario quiere ingresar al sistema &   \\
 2. & El sistema chequea que la plataforma este soportada &   \\
 3. & Se procede con la autenticaci\'on del usuario & Se envia un mensaje al usuario que indica que la plaforma no esta soportada\\
 4. & Fin de caso de uso & \\
\end{tabular}
\end{center}

\bigskip
\item Auditabilidad de votantes
\begin{center}
\begin{tabular}{ll}
Actor & Usuario del rectorado \\
\hline
Pre condición & El usuario se encuentra autenticado\\
\hline
Pos condición & El rectorado sabe que votantes sufragaron \\
\hline
\end{tabular}
\medskip
\begin{tabular}{c p{4cm}|p{4cm}}
 & Curso normal & Curso alternativo \\
 1. & El rectorado selecciona la funcionalidad de auditabilidad &   \\
 2. & El sistema despliega la lista de la facultades disponibles &   \\
 3. & El recortorado elije la facultad que quiere auditar & Se envia un mensaje al usuario que indica que la plaforma no esta soportada\\
 4. & El sistema le envia al rectorado los votantes que sufragaron de la facultad seleccionada \\
 5. & Fin de caso de uso & \\
\end{tabular}
\end{center}



\end{itemize}