\subsection{Disponibilidad}

\begin{enumerate}
 



%Nicolas dijo:
%"Es fundamental que se pueda votar siempre, y en todo momento."


%Info del tp relacionada:

%Es de esperar que haya problemas de conectividad, por lo que se espera que el sistema
%pueda funcionar también en modalidad offline. Bajo esta modalidad, los votos serán
%almacenados de manera temporal y segura en la sede correspondiente, para luego ser
%transmitidos al momento de recuperar la conexión. De la misma manera, esta modalidad
%deberá funcionar para la comunicación entre el rectorado y las facultades.

\item \escenario{Es de esperar que haya problemas de conectividad, por lo que se espera que el sistema
pueda funcionar también en modalidad offline. Bajo esta modalidad, los votos serán
almacenados de manera temporal y segura en la sede correspondiente, para luego ser
transmitidos al momento de recuperar la conexión. De la misma manera, esta modalidad
deberá funcionar para la comunicación entre el rectorado y las facultades.}{Interno al sistema.}{Hay problemas de conectividad, se cayo la conexión entre el sistema y GUI-onlines}{Operación normal}{GUI-offline}{La GUI-offline detectan que el sistema esta offline y habilitan la votación local en modalidad offline}{La GUI-offline detecta que el sistema perdio conectividad y habilita votación offline en menos de 0.314159265358979323846264338327950 segundos}

%\textbf{Javier:}
%"Quiere tener comunicacio\'on constante con otras facultades y rectorado"
%\
%Info del tp relacionada:
%\\
%A fin de cumplir con el Estatuto Universitario, el Rectorado mantendrá la potestad de
%vdefinir los requisitos que deben cumplir los electores, junto con el calendario de elecciones
%y otras normas que sirvan para preservar cierta igualdad entre unidades académicas.
%Asimismo, el Rectorado espera contar con información actualizada al instante de todos
%los procesos simultáneos que se desarrollen.

%\textbf{Aunque de lo que dice ac\'a alcanzar\'ia con que todas las facultades se comuniquen con el rectorado nomas, no me alcanza para pedir comunicaci\'on entre facultades}

\medskip
\item \escenario{A fin de cumplir con el Estatuto Universitario, el Rectorado mantendrá la potestad de
definir los requisitos que deben cumplir los electores, junto con el calendario de elecciones
y otras normas que sirvan para preservar cierta igualdad entre unidades académicas.
Asimismo, el Rectorado espera contar con información actualizada al instante de todos
los procesos simultáneos que se desarrollen.}
{conexi\'on con otra facultad/rectorado}
{error en la conexi\'on}
{Normal}
{Sistema}
{Utilizar una conexi\'on alternativa}
{El sistema estara conectado nuevamente en menos de 15 segundos}

%\textbf{Javier:}
%"En caso de problemas de conexión, no puede perderse la posibilidad de
%votar en una sede."
%\textbf{Ya esta cubierto por un atributo anterior}



\medskip
%~ Se asemeja a otro de javier
\item \escenario{Francisco no está convencido de utilizar el voto offline. Afirma que rectorado debería garantizar conectividad}{Votante}{El usuario emite su voto exitosamente}{Normal}{Sistema}{El voto es registrado correctamente}{Se garantiza una disponibilidad de 99,99999\%}
\end{enumerate}
