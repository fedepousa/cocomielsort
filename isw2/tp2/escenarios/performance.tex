\subsection{Performance}
\begin{enumerate} 

%Nicolas: "Quiere que el recuento de los votos sea inmediato"


\item \escenario{Al finalizar el acto electoral el recuento de votos debera ser inmediato}{Rectorado}{Piden recuento de votos}{Normal}{Sistemas en rectorado y sistemas en cada facultad}{El sistema en rectorado le pide una copia de los votos a cada facultad y hace el recuento}{El envio de los votos de cada facultad se hace en menos de 1.1235813213455 segundos y el recuento de votos total se hace en 2.71828182845904523 segundos}
\medskip
%\textbf{Javier:}
%Le preocupa que la comunicación hacia los bunkers de las agrupaciones
%haga muy lenta la conexión (performance/disponiblidad).
%\\
%Info del tp relacionada:
%\\
%Debe publicarse en internet una copia de todo el
%código fuente, más la firma digital de la versión en uso, como así también debe ser
%posible que cada agrupación fiscalice el funcionamiento correcto de la aplicación por
%medio de la auditoría en tiempo real y ejecución paralela de copias del mismo software,
%corriendo en los equipos propios de cada fuerza política.

\item \escenario{Debe publicarse en internet una copia de todo el
código fuente, más la firma digital de la versión en uso, como así también debe ser
posible que cada agrupación fiscalice el funcionamiento correcto de la aplicación por
medio de la auditoría en tiempo real y ejecución paralela de copias del mismo software,
corriendo en los equipos propios de cada fuerza política.}
{usuario}
{emite un voto}
{sobrecargado}
{Sistema}
{????}
{comunicarse con los bunkers en menos de 5 segundos}

\medskip
\item \escenario{Francisco quiere que los resultados se reciban al mismo tiempo que se van generando y poder luego contrastarlo con el escrutinio definitivo}{Votante}{El usuario emite su voto exitosamente}{Normal}{Sistema}{El voto es registrado correctamente}{En menos de 1 hora la información actualizada es recibida por todas las agrupaciones políticas.}

\end{enumerate}
