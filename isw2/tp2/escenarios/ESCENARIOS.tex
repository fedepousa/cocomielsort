\documentclass[a4paper,11pt]{article}

\usepackage[spanish, activeacute]{babel}

\usepackage{color}
\usepackage{graphicx}
\usepackage[utf8x]{inputenc}
\usepackage{fontenc}
\usepackage{listings}


%comando para armar escenario toma 7 parametros, en orden
%texto chamuyyo fuente estimulo entorno artefacto respuesta medicion
\newcommand\escenario[7]{
	\begin{itemize}
		%texto chamuyo
		\item \textit{#1}  
		
		\item \textbf{Fuente:} #2
		\item \textbf{Estímulo:} #3
		\item \textbf{Entorno:} #4
		\item \textbf{Artefacto:} #5
		\item \textbf{Respuesta:} #6
		\item \textbf{Medición:} #7
		
		
	\end{itemize}
}








\begin{document}

\section{Escenarios}


% ejemplo de uso
% \escenario{Si la comunicación entre una Terminal de Cobro móvil y el Sistema Central se pierde, o los tiempos de transmisión son prohibitivos, la Terminales de Cobro debe de seguir funcionando en modo offline, de forma transparente al usuario.}{Interno al sistema} {Las transacciones de cobro de pasaje en una terminal de cobro están tardando mas de 1 segundo.}{Operación normal.}{Terminal de Cobro.}{La terminal pasa a modo offline, utilizando la información local almacenada en las tarjetas y logueando  toda operación realizada para que este lista cuando la conexión se restablezca.} {No se requieren acciones adicionales por parte del usuario, y no se ve afectada la performance (<= 1 segundo por transacción).}

\subsection{Disponibilidad}

\begin{enumerate}
 



%Nicolas dijo:
%"Es fundamental que se pueda votar siempre, y en todo momento."


%Info del tp relacionada:

%Es de esperar que haya problemas de conectividad, por lo que se espera que el sistema
%pueda funcionar también en modalidad offline. Bajo esta modalidad, los votos serán
%almacenados de manera temporal y segura en la sede correspondiente, para luego ser
%transmitidos al momento de recuperar la conexión. De la misma manera, esta modalidad
%deberá funcionar para la comunicación entre el rectorado y las facultades.

\item \escenario{Se espera que el sistema
pueda funcionar también en modalidad offline.}{Interno al sistema.}{Hay problemas de conectividad, se cay\'o la conexión entre el sistema y GUI-onlines}{Operación normal}{GUI-offline}{Las GUI-offline detectan que el sistema est\'a offline y habilitan la votación local en modalidad offline}{La GUI-offline detecta que el sistema perdi\'o conectividad y habilita votación offline en menos de 0.3 segundos}

%\textbf{Javier:}
%"Quiere tener comunicacio\'on constante con otras facultades y rectorado"
%\
%Info del tp relacionada:
%\\
%A fin de cumplir con el Estatuto Universitario, el Rectorado mantendrá la potestad de
%vdefinir los requisitos que deben cumplir los electores, junto con el calendario de elecciones
%y otras normas que sirvan para preservar cierta igualdad entre unidades académicas.
%Asimismo, el Rectorado espera contar con información actualizada al instante de todos
%los procesos simultáneos que se desarrollen.

%\textbf{Aunque de lo que dice ac\'a alcanzar\'ia con que todas las facultades se comuniquen con el rectorado nomas, no me alcanza para pedir comunicaci\'on entre facultades}

\medskip
\item \escenario{El Rectorado espera contar con información actualizada al instante de todos
los procesos simultáneos que se desarrollen.}
{Conexi\'on con otra facultad/rectorado}
{Error en la conexi\'on}
{Normal}
{Sistema}
{Utilizar una conexi\'on alternativa}
{El sistema estar\'a conectado nuevamente en menos de 15 segundos}

%\textbf{Javier:}
%"En caso de problemas de conexión, no puede perderse la posibilidad de
%votar en una sede."
%\textbf{Ya esta cubierto por un atributo anterior}



\medskip
%~ Se asemeja a otro de javier
\item \escenario{Algunos no est\'an convencidos de utilizar el voto offline y afirman que rectorado debería garantizar conectividad}
{Votante}
{El usuario emite su voto exitosamente}
{Normal}
{Sistema}
{El voto es registrado correctamente}
{Se garantiza una disponibilidad de 99,99999\%}
\end{enumerate}

\newpage

\subsection{Performance}
\begin{enumerate} 

%Nicolas: "Quiere que el recuento de los votos sea inmediato"

\item \escenario{Al finalizar el acto electoral el recuento de votos deber\'a ser inmediato}{Rectorado}{Piden recuento de votos}{Normal}{Sistemas en rectorado y sistemas en cada facultad}{El sistema en rectorado le pide una copia de los votos a cada facultad y hace el recuento}{El env\'io de los votos de cada facultad se hace en menos de 1 segundo y el recuento de votos total se hace en 5 segundos}
\medskip
%\textbf{Javier:}
%Le preocupa que la comunicación hacia los bunkers de las agrupaciones
%haga muy lenta la conexión (performance/disponiblidad).
%\\
%Info del tp relacionada:
%\\
%Debe publicarse en internet una copia de todo el
%código fuente, más la firma digital de la versión en uso, como así también debe ser
%posible que cada agrupación fiscalice el funcionamiento correcto de la aplicación por
%medio de la auditoría en tiempo real y ejecución paralela de copias del mismo software,
%corriendo en los equipos propios de cada fuerza política.

\item \escenario{La auditoría se realiza en tiempo real y ejecución paralela.}
{Usuario} %Partido ?
{Emite un voto}
{Sobrecargado}
{Sistema}
{Se registra el voto y se env\'ia a los fiscalizadores}
{En menos de 5 segundos los bunkers est\'an informados del voto}

\medskip
\item \escenario{Los resultados deben recibirse al mismo tiempo que se van generando para luego ser contrastastados con el escrutinio definitivo}
{Votante}
{El usuario emite su voto exitosamente.}
{Normal}
{Sistema}
{El voto es registrado correctamente.}
{En menos de 1 hora los resultados de votaci\'on son recibidos por todas las agrupaciones políticas.}

\end{enumerate}

\newpage

\subsection{Seguridad}


Nicolas: "Debe bancarse procesos de auditorías y recuentos (seguridad-
auditabilidad)"
\escenario{Debe bancarse procesos de auditorías y recuentos}{Fuente: Pedido de cualquier persona}{Requiere auditar los votos para recontarlos}{Normal}{Sistema}{Le entrega online o en rectorado, una copia en formato digital de los votos emitidos, todo el material estar certificado}{Debe remitir información en menos de una semana}



\newpage

\subsection{Modificabilidad}

%Nicolas: "Quieren usar el sistema también para otro tipo de elecciones, por ejemplo, plebiscitos (modificabilidad/flexibilidad)"

\begin{enumerate}
 
\item   \escenario{Quieren usar el sistema también para otro tipo de elecciones, por ejemplo,
plebiscitos}{Rectorado}{De desea confeccionar un plebiscito}{Normal}{Sistema}{Se configura el sistema para que pueda ofrecer participar en un plescbicito sin ninguna cambio negativo al sistema actual}{Se invierte menos del 12,345678 de las horas empleadas para diseñar el sistema original}

\medskip
%Adrian: Que se pueda votar desde diversas plataformas, celulares, todo!. (extensibilidad, modificabilidad, funcionalidad, flexiblidad)
\item   \escenario{Que se pueda votar desde diversas plataformas, celulares, todo!. (extensibilidad, modificabilidad, funcionalidad, flexiblidad)}{un votante}{Se intenta votar desde una Tablet conectada con 4G por internet.}{Normal}{Sistema web}{El sistema permite votar siendo transparente para este la plataforma}{El sistema resuelve sin tener que ambiar nade de su implementaci\'on}



\end{enumerate}

\newpage


\subsection{Usabilidad}

\begin{enumerate}


\item  \escenario{Se quiere que el sistema sea sencillo de usar, para que no se confunda al electorado al momento de emitir el voto}{Usuario}{El usuario ingresa el voto para el candidato que eligió}{Operación normal}{Sistema}{Se registra el voto del usuario}{El 99,99 \% de los usuarios efectivamente votó al usuario que cree haber votado.}

\medskip

\item  \escenario
{La interfaz debe ser simple, para que no consuma mucho tiempo del usuario}
{Usuario}
{El usuario ingresa el voto para el candidato que eligió}
{Operación normal}
{Sistema}
{Se registra el voto del usuario}
{El 99\% de los usarios logran emitir su voto en dos minutos.}
\medskip
\item  \escenario
{El sistema debera brindar ayuda para aprender a ser usado, de manera que no sea necesario leer un manual}
{Usuario que no sabe usar el sistema}
{El usuario quiere elegir un candidato}
{Operación normal}
{Sistema}
{Se registra el voto del usuario}
{Todos los usuarios logran emitir un voto.}
\medskip

%adrian
\item  \escenario{Interfaz intuitiva, y clara! (usabilidad)}
{Un votante}
{Se ingresa por primera vez al sistema intentando votar}
{Normal}
{Sistema web}
{El sistema permite registrar la votaci\'on}
{El usuario puede registrar su voto en menos de 4 minutos en la primera vez.}

\medskip
\item   \escenario{Usable en otros idiomas!, fácil de configurar el idioma (modificabilidad, usabilidad)}
{Un votante}
{Un votante que domina mal el espa\~nol ingresa el sitio para votar y desea cambiar el idioma a ingl\'es}
{Operación Normal}
{Sistema}
{El sistema permite elegir el idioma y cambiar sus contenidos al idioma correspondiente.}
{El usuario encuentra c\'omo cambiar el idioma en menos de 3 segundos.}
\medskip

\item   \escenario{Desea que se instalen terminales en todas las sedes para que pueda votar todos los alumnos sin conexión a internet.}
{Un votante}
{Un votante sin conexión a internet se acerca a la facultad a emitir su voto en una terminal}
{Normal}
{Terminal de votaci\'on}
{La terminal registra el voto}
{El 100\% de los votantes pueden votar en las terminales de la facultad}

\end{enumerate}


\end{document}
