\section{Casos de Uso}

A continuaci\'on se presentan los principales casos de uso del sistema.
Se muestran las principales interacciones que hay desde el exterior (votantes, rectorado, partidos pol\'iticos) con el sistema.

Notese que existe una herencia de actores para los casos de uso. Los actores Alumno, Graduado y Profesor heredan del actor Usuario externo.


\begin{itemize}
\bigskip
\item Generando contraseña
\bigskip
\begin{center}
\begin{tabular}{ll}
Actor & Usuario externo \\
\hline
Pre condición & Ninguna \\
\hline
Pos condición & El usuario externo posee usuario y contraseña para el sistema\\
\hline
\end{tabular}
\medskip
\\
Detalle: Se dispondrá de una terminal en la sede que le corresponde a cada usuario externo para que pueda generar por primera vez su contraseña personal. Pudiendo chequear así la identidad del usuario externo.
\end{center}

\bigskip
\item Ingresando al sistema
\bigskip
\begin{center}
\begin{tabular}{ll}
Actor & Usuario externo\\
\hline
Pre condición & El usuario externo posee usuario y contraseña \\
\hline
Pos condición & El usuario externo se encuentra autenticado\\
\hline
\end{tabular}
\medskip
\begin{tabular}{c p{4cm}|p{4cm}}
 & Curso normal & Curso alternativo \\
 1. & El usuario externo indica su nombre de usuario y contraseña &  \\
 2. & El sistema valida los datos ingresados & \\
 3. & Si los datos son correctos el usuario externo recibe una confirmación de que su nombre de usuario y contraseña son correctos & Si los datos son incorrectos el usuario externo recibe un mensaje de error indicando que la combinación de nombre de usuario y contraseña es incorrecta \\
 4. & Se muestra la interfaz de voto electrónico para el usuario & Fin del caso de uso \\
 5. & Fin del caso de uso& \\ 
\end{tabular}
\end{center}


\bigskip
\item Postulandose para las elecciones
\bigskip
\begin{center}
\begin{tabular}{ll}
Actor & Usuario externo \\
\hline
Pre condición & El usuario externo se encuentra logueado \\
\hline
Pos condición & El usuario externo se postula para el claustro correspondiente\\
\hline
\end{tabular}
\medskip
\begin{tabular}{c p{4cm}|p{4cm}}
 & Curso normal & Curso alternativo \\
 1. & El usuario externo se postula para su claustro &  \\
 2. & El sistema valida que se cumplan las restricciones para la elección abierta & \\
 3. & Si se cumplen las restricciones se recibe una confirmación de postulación & Si los datos son incorrectos el usuario recibe un mensaje de error\\
 4. & Fin del caso de uso & Fin del caso de uso \\
\end{tabular}
\end{center}

\bigskip
\item Visualizando cantidatos
\bigskip
\begin{center}
\begin{tabular}{ll}
Actor & Usuario externo\\
\hline
Pre condición & El usuario externo se encuentra auntenticado\\
\hline
Pos condición & El usuario externo visualiza los candidatos\\
\hline
\end{tabular}
\\
\medskip
Detalle: Los usuarios externos pueden visualizar a todos los candidatos que se pueden votar en su claustro.
\end{center}

\bigskip
\item Emitiendo voto
\bigskip
\begin{center}
\begin{tabular}{ll}
Actor & Usuario externo \\
\hline
Pre condición & El usuario externo se encuentra visualizando los candidatos y no voto\\
\hline
Pos condición & Se registra el voto del usuario externo\\
\hline
\end{tabular}
\medskip
\begin{tabular}{c p{4cm}|p{4cm}}
 & Curso normal & Curso alternativo \\
 1. & Se muestra ante el usuario la lista de candidatos de su claustro, con una opción para votar en blanco. &  \\
 2. & El usuario selecciona el candidato de su preferencia o votar en blanco & \\
 3. & Se pide confirmación al usuario mostrando claramente qué opción ha elegido & \\
 4. & El votante confirma su elección & Si el votante no confirma la elección en 30 segundos se descarta la selección del usuario \\
 5. & Se muestra el certificado por pantalla & Fin de caso de uso \\
 6. & Fin de caso de uso & \\ 
\end{tabular}
\end{center}




\bigskip
\item Modificando el idioma en la interfaz de usuario
\bigskip
\begin{center}
\begin{tabular}{ll}
Actor & Usuario externo \\
\hline
Pre condición & El usuario externo se encuentra autenticado \\
\hline
Pos condición & La interfaz se presenta en el idioma elegido y se guarda la preferencia\\
\hline
\end{tabular}
\medskip
        \\
Detalle: El usuario externo tiene la posibilidad de elegir el idioma en el que quiere que la interfaz se presente mediante una elección en un botón desplegable.
\end{center}





\bigskip
\item Agregando elección
\begin{center}
\begin{tabular}{ll}
Actor & Rectorado \\
\hline
Pre condici\'on & Ninguna \\
\hline
Pos condici\'on & Se agrega una nueva elección \\
\hline
\end{tabular}
\medskip
\\
Detalle: En este caso de uso el rectorado agrega una nueva elección seteando el calendario, las restricciones, el claustro y la facultad para la misma.
\end{center}


\bigskip
\item Modificando elección
\begin{center}
\begin{tabular}{ll}
Actor & Rectorado \\
\hline
Pre condici\'on & Ninguna \\
\hline
Pos condici\'on & Se modifica una elección aún no finalizada\\
\hline
\end{tabular}
\medskip
\\
Detalle: En este caso de uso el rectorado modifica el calendario de una elección vigente cambiando los períodos, realizando los chequeos necesarios para mantener la validez de la elección.
\end{center}


\bigskip
\item Reglamentando elecciones
\begin{center}
\begin{tabular}{ll}
Actor & Rectorado \\
\hline
Pre condici\'on & Ninguna \\
\hline
Pos condici\'on & Se cambia el reglamento para las elecciones \\
\hline
\end{tabular}
\medskip
\\
Detalle: En este caso de uso el rectorado modifica las restricciones que tienen las elecciones para la postulación de candidatos, para saber qué personas están habilitadas para votar y para indicar cuántos votos por integrante de cada claustro son aceptados.
\end{center}

\bigskip
\item Auditando votantes
\begin{center}
\begin{tabular}{ll}
Actor & Rectorado \\
\hline
Pre condición & Ninguna\\
\hline
Pos condición & El Rectorado sabe qu\'e votantes sufragaron \\
\hline
\end{tabular}
\medskip
\\
Detalle: En este caso de uso se puede ver cómo el Rectorado puede visualizar los votantes que ya sufragaron en una elección dada.
\end{center}  

\bigskip
\item Auditando votantes
\begin{center}
\begin{tabular}{ll}
Actor & Partido político \\
\hline
Pre condición & Ninguna\\
\hline
Pos condición & El Partido sabe qu\'e votantes sufragaron \\
\hline
\end{tabular}
\medskip
\\
Detalle: Caso similar a la auditación por parte del rectorado.
\end{center} 

\bigskip
\item Viendo resultados
\begin{center}
\begin{tabular}{ll}
Actor & Rectorado \\
\hline
Pre condición & Ninguna\\
\hline
Pos condición & El Rectorado visualiza los resultados de una elección \\
\hline
\end{tabular}
\medskip
\\
Detalle: En este caso de uso se puede ver cómo el Rectorado puede visualizar los resultados de una elección finalizada.
\end{center} 

\bigskip
\item Viendo resultados
\begin{center}
\begin{tabular}{ll}
Actor & Partido político \\
\hline
Pre condición & Ninguna\\
\hline
Pos condición & El Rectorado visualiza los resultados de una elección \\
\hline
\end{tabular}
\medskip
\\
Detalle: En este caso de uso se puede ver cómo el Rectorado puede visualizar los resultados de una elección finalizada.
\end{center} 

\bigskip
\item Fiscalizando votaci\'on
\begin{center}
\begin{tabular}{ll}
Actor & Partido Político \\
\hline
Pre condición & El usuario del partido se encuentra visualizando los resultados de la elección\\
\hline
Pos condición & El Partido Político fiscaliza la elección elegida\\
\hline
\end{tabular}
\medskip
\\
Detalle: Un partido político tiene la potestad de fiscalizar cualquier elección finalizada por demanda.
\end{center}


\bigskip
\item Aceptando acta
\begin{center}
\begin{tabular}{ll}
Actor & Rectorado \\
\hline
Pre condición & Se cerró una elección, se generaron los resultados y se están visualizando\\
\hline
Pos condición & Se genera el acta\\
\hline
\end{tabular}
\medskip
\\
Detalle: Cuando se cierra una elección, se generan los resultados que son enviados a Rectorado. Luego un representate tiene que validar los resultados obtenidos para que se genere el acta pertinente.
\end{center}

\bigskip
\item Resolviendo conflictos
\begin{center}
\begin{tabular}{ll}
Actor & Rectorado \\
\hline
Pre condición & Se cerró una elección, se generaron los resultados y se están visualizando\\
\hline
Pos condición & Se genera un acta con los resultados y una enmienda manual\\
\hline
\end{tabular}
\medskip
        \\
Detalle: Cuando se cierra una elección, se generan los resultados que son enviados a Rectorado. Si estos resultados son conflictivos un representante del Rectorado tiene que agregar una enmienda manualmente que aparecerá junto con los resultados automáticos.
\end{center}

\bigskip
\item Visualizando fechas
\begin{center}
\begin{tabular}{ll}
Actor & Rectorado/Partido Político \\
\hline
Pre condición & Hay una elección abierta\\
\hline
Pos condición & El Rectorado/Partido Político visualiza las fechas de la elección elegida\\
\hline
\end{tabular}
\\
\medskip
Detalle: Tanto el Rectorado como los partidos políticos estan habilitados para visualizar las fechas de las elecciones abiertas.
\end{center}


 

\end{itemize}