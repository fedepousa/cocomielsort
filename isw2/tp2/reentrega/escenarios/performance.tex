\subsection{Performance}
\begin{enumerate} 

%Nicolas: "Quiere que el recuento de los votos sea inmediato"

\item \escenario{Al finalizar el acto electoral el recuento de votos deber\'a ser inmediato}{Rectorado}{Piden recuento de votos}{Normal}{Sistemas en rectorado y sistemas en cada facultad}{El sistema en rectorado le pide una copia de los votos a cada facultad y hace el recuento}{El env\'io de los votos de cada facultad se hace en menos de 1 segundo y el recuento de votos total se hace en 5 segundos}
\medskip
%\textbf{Javier:}
%Le preocupa que la comunicación hacia los bunkers de las agrupaciones
%haga muy lenta la conexión (performance/disponiblidad).
%\\
%Info del tp relacionada:
%\\
%Debe publicarse en internet una copia de todo el
%código fuente, más la firma digital de la versión en uso, como así también debe ser
%posible que cada agrupación fiscalice el funcionamiento correcto de la aplicación por
%medio de la auditoría en tiempo real y ejecución paralela de copias del mismo software,
%corriendo en los equipos propios de cada fuerza política.

\item \escenario{La auditoría se realiza en tiempo real y ejecución paralela.}
{Usuario} %Partido ?
{Emite un voto}
{Sobrecargado}
{Sistema}
{Se registra el voto y se env\'ia a los fiscalizadores}
{En menos de 5 segundos los bunkers est\'an informados del voto}

\medskip
\item \escenario{Los resultados deben recibirse al mismo tiempo que se van generando para luego ser contrastastados con el escrutinio definitivo}
{Votante}
{El usuario emite su voto exitosamente.}
{Normal}
{Sistema}
{El voto es registrado correctamente.}
{En menos de 1 hora los resultados de votaci\'on son recibidos por todas las agrupaciones políticas.}

\end{enumerate}
