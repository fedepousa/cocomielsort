\subsection{Usabilidad}

\begin{enumerate}


\item  \escenario{Se quiere que el sistema sea sencillo de usar, para que no se confunda al electorado al momento de emitir el voto}{Usuario}{El usuario ingresa el voto para el candidato que eligió}{Operación normal}{Sistema}{Se registra el voto del usuario}{El 99,99 \% de los usuarios efectivamente votó al usuario que cree haber votado.}

\medskip

\item  \escenario
{La interfaz debe ser simple, para que no consuma mucho tiempo del usuario}
{Usuario}
{El usuario ingresa el voto para el candidato que eligió}
{Operación normal}
{Sistema}
{Se registra el voto del usuario}
{El 99\% de los usarios logran emitir su voto en dos minutos.}
\medskip
\item  \escenario
{El sistema debera brindar ayuda para aprender a ser usado, de manera que no sea necesario leer un manual}
{Usuario que no sabe usar el sistema}
{El usuario quiere elegir un candidato}
{Operación normal}
{Sistema}
{Se registra el voto del usuario}
{Todos los usuarios logran emitir un voto.}
\medskip

%adrian
\item  \escenario{Interfaz intuitiva, y clara! (usabilidad)}
{Un votante}
{Se ingresa por primera vez al sistema intentando votar}
{Normal}
{Sistema web}
{El sistema permite registrar la votaci\'on}
{El usuario puede registrar su voto en menos de 4 minutos en la primera vez.}

\medskip
\item   \escenario{Usable en otros idiomas!, fácil de configurar el idioma (modificabilidad, usabilidad)}
{Un votante}
{Un votante que domina mal el espa\~nol ingresa el sitio para votar y desea cambiar el idioma a ingl\'es}
{Operación Normal}
{Sistema}
{El sistema permite elegir el idioma y cambiar sus contenidos al idioma correspondiente.}
{El usuario encuentra c\'omo cambiar el idioma en menos de 3 segundos.}
\medskip

\item   \escenario{Desea que se instalen terminales en todas las sedes para que pueda votar todos los alumnos sin conexión a internet.}
{Un votante}
{Un votante sin conexión a internet se acerca a la facultad a emitir su voto en una terminal}
{Normal}
{Terminal de votaci\'on}
{La terminal registra el voto}
{El 100\% de los votantes pueden votar en las terminales de la facultad}

\end{enumerate}
