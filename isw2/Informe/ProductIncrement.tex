\section{Product Increment}

El primer paso para realizar la demo fue elegir el lenguaje que utilizaremos.
Decidimos utilzar Smalltalk debido a que habia miembros del equipo que estaban familiarizados con las herramientas para desarrollo 
y porque al ser un lenguaje de objetos resulta m\'as natural modelar al mundo e implementar un sistema a partir de un dise\~no de objetos.

\medskip

Las funcionalidades a implementar solicitadas por el enunciado son:
\begin{itemize}
 \item Alta de candidatos
 \item Emisi\'on de voto de un miembro del departamento
 \item Confecci\'on del acta final resultante de la elecci\'on
\end{itemize}

\medskip

La demo consta de un paquete de Smalltalk que permite realizar las elecciones.
La interacci\'on la realizamos mediante el transcript. Incialmente cargamos padrones de alumnos, profesores y graduados y creamos instancias de objetos
que sirven como validadores de candidatos. Luego tenemos un objeto llamado elecci\'on que es el que ser\'a encargado de todo el proceso, es decir,
para utilizar el sistema lo haremos mediante este objeto.

Otro objeto con el que interactuaremos sera el logeador, que permitira autenticar
a los usuarios para realizar las distintas acciones que seran enviadas por medio de la interfaz correspondiente al periodo de elecci\'on en la que se
encuentre la elecci\'on.
