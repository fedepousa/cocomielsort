\section{Product Backlog}


A continuaci\'on se detallan las Users Stories pertenecientes al presente proyecto.
Las mismas se clasifican seg\'un las diferentes funcionalidades atacadas del sistema, divisi\'on que luego resultar\'a importante tanto para el dise\~{n}o del sistema, como para la implementaci\'on del mismo.

\subsection{Logueo al sistema}


\bigskip

\begin{tabular}{|l|}
\hline
Como: Usuario\\
\hline
Quiero: Poder loguearme al sistema\\
\hline
De forma que: Pueda utilizar las diferentes opciones del mismo, como la votaci\'on. \\
\hline
\end{tabular}

\bigskip

\begin{tabular}{|l|}
\hline
Como: Usuario\\
\hline
Quiero: Que nadie se loguee con mi cuenta\\
\hline
De forma que: Nadie pueda candidatearme o votar por mi. \\
\hline
\end{tabular}



\bigskip

\begin{tabular}{|l|}
\hline
Como: Usuario\\
\hline
Quiero: Poder desloguearme de forma segura del sistema\\
\hline
De forma que: Nadie pueda utilizar mis datos o mi cuenta.\\
\hline
\end{tabular}


\bigskip


\subsection{Postularse como candidato}

\begin{tabular}{|l|}
\hline
Como: Junta electoral\\
\hline
Quiero: Poder abrir etapa de postulaciones\\
\hline
De forma que: los usuarios interesados puedan candidatearse \\
\hline
\end{tabular}

\bigskip

\begin{tabular}{|l|}
\hline
Como: Aspirante a candidato\\
\hline
Quiero: Poder postularme como candidato\\
\hline
De forma que: Pueda presentarme en las siguientes elecciones \\
\hline
\end{tabular}


\bigskip

\begin{tabular}{|l|}
\hline
Como: Candidato\\
\hline
Quiero: Poder darme de baja\\
\hline
De forma que: Pueda arrenpetirme de postularme como candidato \\
\hline
\end{tabular}


\bigskip

\begin{tabular}{|l|}
\hline
Como: Junta electoral\\
\hline
Quiero: Que el proceso de postulaci\'ion chequee los requerimientos del postulado\\
\hline
De forma que: Solo puedan postularse las personas que cumplan con los requisitos establecidos \\
\hline
\end{tabular}


\bigskip

\begin{tabular}{|l|}
\hline
Como: Junta electoral\\
\hline
Quiero: Poder cambiar el reglamento de postulaci\'on\\
\hline
De forma que: Los requerimientos puedan ser adaptables \\
\hline
\end{tabular}


\bigskip

\subsection{Emisi\'on de voto}


\begin{tabular}{|l|}
\hline
Como: Junta electoral\\
\hline
Quiero: Poder abrir etapa de votaci\'on\\
\hline
De forma que: se termine la etapa de postulaci\'on y comience la votaci\'on de los candidatos\\
\hline
\end{tabular}

\bigskip

\begin{tabular}{|l|}
\hline
Como: Usuario\\
\hline
Quiero: Poder emitir mi voto\\
\hline
De forma que: mi voto quede registrado para la votaci\'on actual.\\
\hline
\end{tabular}


\bigskip

\begin{tabular}{|l|}
\hline
Como: Junta electoral\\
\hline
Quiero: Saber quien voto\\
\hline
De forma que: Pueda ir teniendo idea de que porcentaje de empadronados se presentaron\\
\hline
\end{tabular}


\bigskip

\begin{tabular}{|l|}
\hline
Como: Usuario\\
\hline
Quiero: Recibir un certificado de emisi\'on de voto\\
\hline
De forma que: tenga una forma de demostrar mi participaci\'on en la votaci\'on\\
\hline
\end{tabular}


\bigskip

\begin{tabular}{|l|}
\hline
Como: Usuario\\
\hline
Quiero: Que mi voto sea an\'onimo\\
\hline
De forma que:  \\
\hline
\end{tabular}


\bigskip

\subsection{Finalizaci\'on del comicio}

\begin{tabular}{|l|}
\hline
Como: Junta electoral\\
\hline
Quiero: Poder cerrar el per\'iodo de votaci\'on\\
\hline
De forma que: se de paso al escrutinio de los votos\\
\hline
\end{tabular}


\bigskip

\begin{tabular}{|l|}
\hline
Como: Junta electoral\\
\hline
Quiero: que se labre un acta con los resultados\\
\hline
De forma que: se obtenga una presentaci\'on formal de los resultados de la votaci\'on \\
\hline
\end{tabular}


\bigskip

\begin{tabular}{|l|}
\hline
Como: Usuario\\
\hline
Quiero: Poder loguearme al sistema\\
\hline
De forma que: Pueda utilizar las diferentes opciones del mismo, como la votaci\'on. \\
\hline
\end{tabular}


\bigskip

\begin{tabular}{|l|}
\hline
Como: Junta Electoral\\
\hline
Quiero: Ser notificada de los conflictos en la designaci\'on de cargos\\
\hline
De forma que: se puedan resolver mediante mecanismos ad hoc \\
\hline
\end{tabular}
