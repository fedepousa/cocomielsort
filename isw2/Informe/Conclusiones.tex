\section{Conclusiones}

Luego de realizar el tp, el grupo lleg\'o a varias conclusiones que son presentadas a continuaci\'on.


\subsection{SCRUM}
La metodolog\'ia de scrum parece ser una muy buena idea para implementar proyectos de magnitud acotada, en los cuales resulta interesante ir teniendo versiones del sistema con funcionalidades pequeñas, que luego son incrementadas en cada iteraci\'on del proceso.

Sin embargo, nos parece que las virtudes de esta metodolog\'ia no emergieron durante la realizaci\'on del presente trabajo pr\'actico, no por un problema de la metodolog\'ia, sino por una incompatibilidad de la misma con la naturaleza de un trabajo pr\'actico. Por nombrar un ejemplo, es muy dificil querer realizar un sprint en donde las personas que forman parte del mismo no manejan los mismos horarios para dedicarse al mismo, y donde adem\'as esta dedicaci\'on fluct\'ua considerablemente dependiendo de otras actividades.

\subsection{Diseño orientado a objetos}

Como conclusi\'on sobre el diseño de objetos, podemos destacar que nos parece que hacer un buen diseño tiene varias ventajas para modelar la realidad de una forma correcta. Entre las ventajas claras se puede mencionar la simplicidad de la implementaci\'on una vez que el diseño esta bien definido, as\'i como tambi\'en se destaca el hecho de que hacer un buen diseño nos ayuda a que el sistema tenga extensibilidad donde nosotros queramos, ya que justamente se diseño con ese preconcepto.

Por otro lado, nos parece que hacer un buen diseño de objetos no es algo facil de aprender y, auguramos, debe ser una habilidad que se obtiene m\'as de la pr\'actica misma que de la adquisici\'on de nuevos conceptos. La desventaja que le vemos a este tipo de diseños proviene justamente de esta dificultad para generar un buen diseño con poca pr\'actica, ya que al no poder discernir entre las bondades de diferentes diseños, se utiliz\'o demasiado tiempo iterando sobre el diseño mismo, lo cual funcionaba como barrera para todas las dem\'as tareas para realizar en el tp.


\subsection{Diagramas UML}

En varias oportunidades durante el tp realizamos diagramas de clase, objetos y secuencia, parecidos a los estandarizados por el UML. Esto nos gener\'o algunos problemas debido a que para ninguno de ellos ten\'iamos una formaci\'on s\'olida de como elaborarlos. Sin embargo, nos parece que esto ayuda al dinamismo de la creaci\'on de los mismos para no centrar la complejidad en la notaci\'on, y si centrarla en el diseño mismo.

