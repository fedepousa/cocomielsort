\section{Introducci\'on}

La idea del presente trabajo pr\'actico es realizar el diseño y la implementaci\'on correspondiente a un sprint de Scrum para un sistema de elecciones de Codep.

Para realizar esta tarea, se simular\'a un sprint de scrum, generando primero las users stories que corresponden al backlog de todo el producto, y luego seleccionando las users stories que pertenecen a las funcionalidades que se deben presentar en esta entrega. 

Una vez generadas las users stories a tener en cuenta, se desarroll\'o el diseño total del sistema, abarcando todas las funcionalidades que se quieren tener en cuenta en este primer sprint.

Para realizar el diseño total del sistema, se fueron atacando cada funcionalidad por s\'i sola, y luego se fue combinando todo el diseño para tener una unificaci\'on final del sistema.

Luego, para reforzar el diseño de clases, se fueron desarrollando diferentes diagramas de clase y de objetos para entender mejor las relaciones tanto din\'amicas como est\'aticas de los objetos relativos a las diferentes funcionalidades.


Por \'ultimo, se implemento las funcionalidades deseadas en lenguaje Smalltalk.