\section{Sprint Backlog}

En el sprint deber\'ian ir las users stories que se van a atacar durante cierto per\'iodo de tiempo. 
Este per\'iodo de tiempo no se encuentra prefijado como en un ambiente de desarrollo, sino que depende de las fechas de entrega del trabajo pr\'actico, sumado al hecho que la dedicaci\'on no es constante y peri\'odica como lo ser\'ia en un ambiente de desarrollo.

\bigskip

Dado que as\'i lo requiere el enunciado, que en este caso cumplir\'ia el rol de un product owner, el presente sprint solamente contiene la users stories correspondientes a las funcionalidades que se quieren atacar en esta iteraci\'on. Por lo tanto, para este sprint se tuvieron en cuenta la users stories correspondientes a las funcionalidades de:

\bigskip

\subsection*{Logging}

\textsl{Como usuario quiero poder logear al sistema de manera que pueda utilizar las diferentes opciones del mismo, como la votaci\'on.}
%2 50
\begin{itemize}
 \item Armar funcionalidad para logear
 \item Entrar al sistema logeado
\end{itemize}

\medskip

\textsl{Como usario quiero que nadie se logee a mi cuenta de manera que nadie pueda candidatearme a mi o votar por mi.}
%3 100
\begin{itemize}
 \item Autenticaci\'on de constrase\~nas mediante un hash
 \item Mensaje de error en caso de contrase\~na incorrecta.
\end{itemize}
\bigskip

\subsection*{Postulaci\'on de candidatos}

\textsl{Como junta electoral quiero poder abrir la etapa de postulaciones de forma que los usuarios interesados puedan candidatearse}
%5 20
\begin{itemize}
 \item Armar funcionalidad para iniciar etapa de postulaciones
 \item Verificar que el usuario corresponde a un miembro de la junta electoral
\end{itemize}

\medskip

\textsl{Como aspirante a candidato quiero poder posultarme como candidato de forma que pueda presentarme en las siguientes elecciones}
%2 20
\begin{itemize}
 \item Armar funcionalidad para que el aspirante pueda candidatearse
 \item Registrar un nuevo candidato
 \item Testear el agregado de candidatos
\end{itemize}

\medskip

%5 50
\textsl{Como junta electoral quiero que el proceso de postulaci\'on chequee los requerimientos del postulado de forma que solo puedan postularse las personas que cumplan con los requerimientos establecidos}
\begin{itemize}
 \item Validar el candidato utilizando las reglas para validar dicho candidato.
 \item Armar ventana de aviso si el candidato es inv\'alido.
\end{itemize}

\medskip
%8 75
\textsl{Como junta electoral quiero poder cambiar el reglamento de postulaci\'on de forma que los requerimientos puedan ser adaptables}
\begin{itemize}
 \item Crear validador de candidatos para cada claustro.
 \item Permitir que un usuario de junta electoral pueda cambiar el validador.
\end{itemize}

\bigskip

\subsection*{Votaci\'on}

\textsl{Como junta electoral quiero poder abrir la etapa de votaci\'on de forma que se termine la etapa de postualaci\'on y comienze la votaci\'on de candidatos}
%5 100
\begin{itemize}
 \item Armar funcionalidad para cerrar postulaciones
 \item Armar funcionalidad para iniciar el inicio de los comicios
 \item Colectar los candidatos que fueron correctamente validados 
 \item Verificar que el usuario es miembro de la junta electoral
 \item Testear el cierre de postulaciones
\end{itemize}

\medskip

\textsl{Como usuario quiero poder emitir mi voto de forma que mi voto quede registrado para la votaci\'on actual}
%1 50
\begin{itemize}
 \item Verficar que el usuario no haya votado
 \item Verificar que el candidato no pueda votar a miembros no candidatos de su claustro
 \item Registrar el voto del usuario
\end{itemize}

\bigskip

\subsection*{Clausura del acto electoral y generaci\'on de resultados}

\textsl{Como junta electoral quiero poder cerrar el per\'iodo de votaci\'on de forma que se de paso al escrutinio de votos}
%3 20
\begin{itemize}
 \item Funcionalidad para cerrar los comicios
 \item Realizar el armado del escrutino por cada claustro
\end{itemize}

\medskip

\textsl{Como junta electoral quiero que se labre un acta con los resultados de forma que se obtenga una presentaci\'on formal de los resultados de la votaci\'on}
%3 100
\begin{itemize}
 \item Decidir los candidatos que resutaron electos.
 \item Armar el acta con los resultados
 \item Testear funcionalidad del armado de acta
\end{itemize}

\bigskip
  


