\documentclass[a4paper,10pt]{article}
\usepackage{graphicx}
\usepackage{verbatim}
% \usepackage{lstlisting}
\usepackage{subfig}
\usepackage{float}
 \usepackage[spanish]{babel}   %ver bien como es
\usepackage[utf8]{inputenc}


\begin{document}

\tableofcontents

\newpage


\begin{center}
\section*{Aclaraciones Generales}
\addcontentsline{toc}{section}{Aclaraciones Generales} 

\begin{itemize}
\item La implementación de todos los algoritmos se realizó en lenguaje C++.

\item Para calcular los tiempos de ejecución de los algoritmos se utilizó la función gettimeofday(), que se encuentra en la librería $<sys/time.h>$. Dado que dicha función funciona solamente en sistemas operativos de tipo linux, se debe compilar con el flag -DTIEMPOS en este tipo de sistemas para poder hacer uso de las mismas.

\item Para la realización de los gráficos se utilizó Qtiplot
\end{itemize}

\end{center}

\newpage

\begin{center}
\section*{Situaciones de la vida real que se pueden modelar utilizando MAX-SAT}
\addcontentsline{toc}{section}{Situaciones de la vida real que se pueden modelar utilizando MAX-SAT} 
\end{center}


\begin{center}
\section*{Algoritmo exacto para MAX-SAT}
\addcontentsline{toc}{section}{Algoritmo exacto para MAX-SAT} 
\end{center}


\begin{center}
\section*{Heur\'istica constructiva para MAX-SAT}
\addcontentsline{toc}{section}{Heur\'istica constructiva para MAX-SAT} 
\end{center}


\begin{center}
\section*{Heur\'istica de b\'usqueda para MAX-SAT}
\addcontentsline{toc}{section}{Heur\'istica de b\'usqueda para MAX-SAT} 
\end{center}


\begin{center}
\section*{Metaheur\'istica de b\'usqueda tab\'u [1,2] para MAX-SAT}
\addcontentsline{toc}{section}{Metaheur\'istica de b\'usqueda tab\'u [1,2] para MAX-SAT} 
\end{center}

\end{document}
