\documentclass[a4paper,10pt]{article}
\usepackage{graphicx}
\usepackage{verbatim}
% \usepackage{lstlisting}
\usepackage{subfig}
\usepackage{float}
 \usepackage[spanish]{babel}   %ver bien como es
\usepackage[utf8]{inputenc}


\begin{document}

\tableofcontents

\newpage


\begin{center}
\section*{Aclaraciones Generales}
\addcontentsline{toc}{section}{Aclaraciones Generales} 

\begin{itemize}
\item La implementación de todos los algoritmos se realizó en lenguaje C++.

\item Para calcular los tiempos de ejecución de los algoritmos se utilizó la función gettimeofday(), que se encuentra en la librería $<sys/time.h>$. Dado que dicha función funciona solamente en sistemas operativos de tipo linux, se debe compilar con el flag -DTIEMPOS en este tipo de sistemas para poder hacer uso de las mismas.

\item Para la realización de los gráficos se utilizó Qtiplot
\end{itemize}

\end{center}

\newpage

\section*{Introducci\'on}
\addcontentsline{toc}{section}{Introducci\'on}

En el presente trabajo se busc\'o realizar diferentes aproximaciones a la resoluci\'on del problema MAX-SAT. El problema MAX-SAT es un problema de optimizaci\'on proveniente del problema de decisi\'on SAT. 

El problema SAT se basa es decidir si un conjunto de clausulas en forma normal conjuntiva, tiene alguna asignaci\'on de las variables que las componenen, tal que la evaluaci\'on de todas las clasulas sea verdadera con dicha asignaci\'on.

El problema SAT es un problema muy importante dentro del campo de la teoria de la complejidad, esto se debe a que SAT fue el primer problema que se identific\'o como NP-Completo. El Teorema de Cook demuestra que el algoritmo SAT pertenece a esta clase de algoritmos.

La importancia de este algoritmo no radica solamente en haber sido el primero en ser caracterizado como NP-Completo, se demostr\'o que el problema SAT puede ser reducido al problema 3-SAT, que es basicamente el mismo problema pero en el cual todas las clausulas tienen un m\'aximo de 3 literales. Adem\'as de probar la reducci\'on, se demostro que este problema tambi\'en pertenece a la clase NP-Completo (A diferencia del problema 2-SAT, para el cual se conoce un algoritmo polinomial para resolverlo). Esta reducci\'on del problema a 3-SAT es un resultado importante ya que luego para probar que otros problemas se encuentran tambien en esta clase se utilizaron reducciones a 3-SAT mostrando la equivalencia en cuanto a la complejidad de resoluci\'on.

\section*{Situaciones de la vida real que se pueden modelar utilizando MAX-SAT}
\addcontentsline{toc}{section}{Situaciones de la vida real que se pueden modelar utilizando MAX-SAT} 




\section*{Algoritmo exacto para MAX-SAT}
\addcontentsline{toc}{section}{Algoritmo exacto para MAX-SAT} 
Como su nombre lo indica, el algoritmo exacto para Max-Sat se encarga de resolver el problema exactamente, arrojando la asignacion que valida la mayor cantidad de clausulas posibles. Dado que no se conoce ning\'un algoritmo polinomial para resolver este problema, se implementaron 2 algoritmos de complejidad exponencial. Por un lado se implement\'o un algoritmo de fuerza bruta de simple implementaci\'on pero de muy baja eficiencia, en cuanto a tiempo de ejecuci\'on. Por otro lado se implemento un algoritmo exacto mediante backtracking para poder evitar visitar todas las asignaciones de las variables posibles.

\subsection*{Algoritmo de fuerza bruta}
\addcontentsline{toc}{subsection}{Algoritmo de fuerza bruta}
En este algoritmo la idea es muy simple, se generan absolutamente todas las asignaciones posibles que existen, siendo estas $2^{v}$ donde v es la cantidad de variables. Luego, por cada una de las asignaciones se verifica cuantas clausulas valida, en el momento que una asignaci\'on supera el m\'aximo de clausulas hasta el momento, se actualiza la cantidad de clausulas validadas, as\'i como cual es la asignaci\'on que gener\'o este m\'aximo.

La idea de este algoritmo es tener una resoluci\'on muy simple del problema, es claro que el tiempo de ejecuci\'on va a ser muy malo ya que se revisan todas y cada una de las asignaciones posibles, y estas crecen en orden exponencial en funci\'on de la cantidad de variables. Sin embargo, cabe destacar que el algoritmo provee una resoluci\'on exacta del problema y con baja probabilidad de errores dada la simpleza del mismo.

A continuaci\'on se presenta el pseudoc\'odigo del mismo:

\begin{verbatim}
maxSatExacto(Vector clausulas, int variables)
vector asignacion
int max := 0
inicializar asignacion todos en falso
Para i = 1 hasta 2^variables
    int sat := 0
    Para j = 1 hasta tamanio(clausulas)
       Si haceTrue(asignacion, clausulas[j])
          sat:= sat + 1
       fin si
    fin para
    si sat > max
        actualizar max
        actualizar asignacionMax
    fin si
    asignacion := siguiente(asignacion,i+1)
fin para
devolver asignacionMax, max
\end{verbatim}

Lo que muestra el pseudoc\'odigo anterior es como, por cada asignaci\'on posible, se mira cada clausula y si la funci\'on \emph{haceTrue} devuelve true, entonces se suma 1 a la cantidad de satisfechas por esa asignaci\'on. Por \'ultimo, se mira cual asignaci\'on es la que tiene m\'as clausulas satisfechas.

La funci\'on \emph{haceTrue} lo que hace es simplemente mirar toda la clausula pasada como par\'ametro y ver si algun literal esta asignado como verdadera, cuando encuentra uno deja de buscar y devuelve True. En caso contrario, si llega hasta el final de la clausula, devuelve False. 

Por otro lado, la funci\'on \emph{siguiente} se encarga de modificar para la asignaci\'on para probar con todas las posibles.

\subsection*{Algoritmo de backtracking}
\addcontentsline{toc}{subsection}{Algoritmo de backtracking}

Luego de implementar un algoritmo exacto por fuerza bruta, se busc\'o implementar un algoritmo tambi\'en exacto pero tratando de lograr un menor tiempo de ejecuci\'on. El algoritmo implementado es un algoritmo exacto basado en la t\'ecnica de backtracking para lograr mejores resultados (en cuanto a tiempo de ejecuci\'on), si bien en la siguiente secci\'on se ver\'a que la complejidad temporal es la misma para ambos algoritmos exactos, la t\'ecnica de backtracking provee de herramientas para no tener que consultar necesariamente por cada una de las asignaciones posibles, es en estas podas que este algoritmo mejora los tiempos de ejecuci\'on del anterior.

\bigskip

La idea de este algoritmo es la siguiente: se genera un arbol de asignaciones, donde cada nivel del arbol i, representa todas las asignaciones posibles desde la variable 1 hasta la variable i. Este arbol, es un arbol binario dado que se arranca de la asignacion nula y de all\'i se abren dos caminos, asignarle False a la variable 1, o asignarle True. Luego, cada rama se va bifurcando sucesivamente por cada variable nueva. Como se puede ver, en el peor caso que tengamos que recorrer todo el arbol la cantidad de asignaciones nuevamente esta dada por $2^{v}$ al igual que en el algoritmo exacto.

Una vez que se tiene el arbol de backtracking lo que se hace es comenzar a recorrer el arbol en alguna direcci\'on determinada. Cabe destacar que en el algoritmo implementado siempre se recorre primero la rama correspondiente a asignar falso a la variable y luego la otra.

La mejora del algoritmo radica en no recorrer todas las ramas posibles, esto se realiza de la siguiente manera: Al haber recorrido la primer rama del arbol llegando hasta una hoja, ya se tiene una mejor soluci\'on posible. Luego, cuando se este explorando una rama lo que se hace es fijarse si esa rama ya posee m\'as clausulas insatisfechas que la mejor soluci\'on hasta el momento. En caso afirmativo, la rama ya no sirve ya que no se podra mejorar la soluci\'on y entonces se puede descartar todo el subarbol que pende de esa rama. Entonces se realiza el backtracking para ir por otro camino posible. 

Las principales diferencias con el algoritmo de fuerza bruta son:
\begin{itemize}
\item La implementaci\'on es bastante m\'as complicada ya que como se realiza backtracking, se debe guardar los estados intermedios de toda la rama que se esta analizando para poder volver hacia atras y tomar un nuevo camino. En el algoritmo exacto, cada asignacion se contrasta con las clausulas originales por lo que solo se deben guardar una vez todas las clausulas.
\item El tiempo de ejecuci\'on deberia ser en la mayor\'ia de los casos. Si bien el peor caso no cambiar\'ia es importante destacar que las podas realizadas pueden traer grandes beneficios en cuanto al tiempo de ejecuci\'on. Si, por ejemplo, ya se tuviese una soluci\'on donde n clausulas son insatisfechas y al asignar True a la variable 1, n+1 clausulas se vuelven insatisfacibles entonces se podr\'ia podar toda una mitad del arbol.
\item El algoritmo de fuerza bruta no puede ser influenciado por alguna heur\'istica, mientras que el algoritmo con backtraking si. Lo que se quiere notar con esto, es que por m\'as que ya se tenga una solucion con n clausulas insatisfechas, el algoritmo por fuerza bruta tiene que probar todas las asignaciones posibles; mientras que el algoritmo de backtracking ya puede comenzar podando ramas que tengan m\'as de n clausulas insatisfechas.
\end{itemize}

A continuaci\'on se presenta el pseudoc\'odigo del algoritmo exacto con backtracking:

\begin{verbatim}
pseudo de backtracking
\end{verbatim}

\subsection*{Complejidad de algoritmos exactos}
\addcontentsline{toc}{subsection}{Complejidad de algoritmos exactos}

En primer lugar, se analizar\'a la complejidad del algoritmo realizado por fuerza bruta.

El mismo presenta un ciclo que se realiza $2^{v}$ veces, dado que esa es la cantidad total de asignaciones diferentes, siendo v la cantidad de variables. Una vez dentro de este ciclo, encontramos otro ciclo que itera sobre la totalidad de las clausulas, haciendo que este ciclo se realice c veces. 

Luego, dentro de este ciclo se llama a la funcion haceTrue. Esta funci\'on lo que hace es fijarse en todos los literales de la clausula si se encuentra asignado como verdadero. Dado que el hecho de fijarse es O(1), ya que es mirar un array, y dado que la clausula como mucho tiene 2*v literales (todas las variables negadas y sin negar), se puede ver que esta funci\'on es O(v).

Por \'ultimo, las otras operaciones que se realizan dentro del ciclo principal tienen menor complejidad que lo mostrado anteriormente ya que son asignaciones que toman O(1) o es la funcion siguiente que toma O(v).

Por lo mostrado anteriormente resulta que la complejidad temporal del algoritmo es O($(2^{v})$*c*v)
Como se puede ver, lo importante m\'as alla de la complejidad exacta, es que este algoritmo es exponencial en funci\'on de la cantidad de variables. Cabe destacar que si bien se podri\'an encontrar algoritmos con mejor complejidad, esta no podr\'ia ser menor que exponencial en el caso que se quieran revisar todas las asignaciones posibles.

\bigskip

Complejidad del algoritmo exacto con backtracking:

Para realizar el c\'alculo de complejidad de este algoritmo tomaremos el peor caso posible. El mismo consiste en que ninguna rama sea podada y por lo tanto se tengan que revisar todos los nodos posibles del arbol. En este caso el ciclo principal que itera sobre todas las asignaciones posibles ser\'ia O($2^{v}$), ya que este es el orden de la cantidad de nodos del arbol. 

Luego, hay que analizar cuales son las operaciones que se realizan cada vez que el algoritmo se encuentra en un nuevo nodo. En primer lugar, se llama a la funci\'on resolver, la misma tiene un funcionamiento an\'alogo al explicado en el algoritmo por fuerza bruta; se fija en cada clausula si la misma se hizo verdadera, y cuenta las insatisfacibles, por lo que esta funcion toma O(c*v) operaciones. Luego, lo que se hace es copiar todo el estado al siguiente nodo para que se pueda procesar, al realizar esto se copian varios par\'ametros. Sin embargo, el orden temporal de esta copia esta dado por el orden que toma copiar todas las clasulas, dado que los dem\'as par\'ametros que se copian tienen menor orden espacial ya que son solamente vectores. 

Se puede ver que copiar todo el estado toma O(c*v) operaciones ya que lo que se hace es copiar absolutamente todas las clausulas, donde cada una pueda tener hasta 2*v literales por lo justificado anteriormente.

Por \'ultimo, resumiendo lo anteriormente explicado se puede ver que cuando se revisa un nodo se toma O(c*v) operaciones, y en el peor caso hay que revisar O($2^{v}$) nodos; por lo que la complejidad total de este algoritmo es O($(2^{v})$*c*v).

Como se puede ver, la complejidad del algoritmo de backtracking no es mejor que el algoritmo exacto por fuerza bruta, sigue siendo exponencial. De hecho, se puede notar que en el peor caso posible, el algoritmo de backtracking es m\'as lento que el algoritmo por fuerza bruta dado el overhead que produce copiar todo el estado al siguiente nodo del arbol para que se pueda procesar.

Sin embargo, en la mayor\'ia de los casos, el algoritmo con backtracking presenta resultados temporalmente mejores que el algoritmo de fuerza bruta ya que se podan varias ramas haciendo que no se tenga que visitar todas las asignaciones posibles.

\medskip

Cabe destacar que la ventaja temporal que gana el algoritmo de backtracking, genera una mayor complejidad espacial ya que en todo momento se debe tener todos los estados de los nodos de la rama que se esta visitando para poder caminar hacia atras. Luego la complejidad espacial del algoritmo de fuerza bruta es O(c*v) ya que se guardan una vez todas las clausulas, mientras que la complejidad espacial del algoritmo de backtracking es O(c*v*v) ya que se guardan todas las clausulas una vez por cada nodo de la rama que se esta analizando, que a lo sumo tiene v nodos.




\section*{Heur\'istica constructiva para MAX-SAT}
\addcontentsline{toc}{section}{Heur\'istica constructiva para MAX-SAT} 
En esta secci\'on se explicar\'a el uso de una heur\'istica constructiva para resolver el problema.

La motivaci\'on principal de utilizar diferentes heur\'isticas para resolver el problema de Max-Sat es que, como se vio en la secci\'on anterior, los algoritmos exactos conocidos son de orden exponencial por lo que solo se pueden correr con instancias relativamente peque\~{n}as.

La idea entonces es realizar un algoritmo que no sea exacto, sino que arroje una soluci\'on aproximada, pero en un tiempo polinomial para poder correr instancias m\'as grandes.

En esta primer aproximaci\'on, se realiz\'o una heur\'istica constructiva, esto quiere decir que se va construyendo una soluci\'on mediante alg\'un criterio que se supone (y luego se ver\'a que se puede demostrar) que es adecuado para obtener un buen resultado.

En este caso, la heur\'istica constructiva seguir\'a una estrategia golosa para ir armando la soluci\'on. Lo que hace esta heur\'istica es revisar todas las clausulas buscando cual es el literal que m\'as se repite. Una vez encontrado dicho literal lo que se hace es asignar True a este literal y actualizar las clausulas en base a esta asignaci\'on.

Actualizar las clausulas consiste en dos tareas, en primer lugar se borran todas las clausulas que contenian este literal ya que ya fueron satisfechas mediante la asignaci\'on y se actualiza el contador de clausulas satisfechas. Por otro lado, en las clausulas restantes, se borra el literal negado ya que este no sirve para futuras elecciones porque al asignar True al literal elegido, el literal negado tendra necesariamente valor falso.

Luego de realizar esta actualizaci\'on, se continua iterativamente con esta estrategia golosa hasta encontrar una asignaci\'on completa de las variables.

A continuaci\'on se muestra el pseudoc\'odigo de la heur\'istica constructiva:

\begin{verbatim}
Constructiva(Vector clausulas, int V)
Max := 0
Comenzar con asignacion vacia
mientras asignacion no este completa
      Tomar literal l que mas se repita
      Asignar True a l
      Para i de 1 hasta tamanio(clausulas)
         Si esta(l, clausula[i])
             borrar clausula[i]
             Max := Max + 1
         fin si
      fin para
      Para i de 1 hasta tamanio(clausulas)
         Si esta(-l, clausula[i])
             borrar(-l, clausula[i])
         fin si
      fin para
fin mientras
devolver asignacion y Max
\end{verbatim}




\section*{Heur\'istica de b\'usqueda local para MAX-SAT}
\addcontentsline{toc}{section}{Heur\'istica de b\'usqueda local para MAX-SAT} 



\section*{Metaheur\'istica de b\'usqueda tab\'u para MAX-SAT}
\addcontentsline{toc}{section}{Metaheur\'istica de b\'usqueda tab\'u para MAX-SAT} 


\end{document}
