\documentclass[a4paper,10pt]{article}
\usepackage{graphicx}
\usepackage{verbatim}
\usepackage{subfig}
\usepackage{float}
\usepackage[spanish]{babel}   %ver bien como es
\usepackage[utf8]{inputenc}

\begin{document}

\tableofcontents

\newpage


\section*{Introducci\'on}
\addcontentsline{toc}{section}{Introducci\'on}

Cosas que dijo la catedra:

acá debería figurar todo lo necesario como para que un
lector no iniciado en el tema entienda el propósito general del
sistema que van a describir. Puede citar algunos fragmentos del
enunciado si lo consideran necesario, pero NO DEBE SER una copia del
enunciado.

\section*{Presunciones}
\addcontentsline{toc}{section}{Presunciones}

Cosas que dijo la catedra:

acá deberían listar aquellas cuestiones que asumieron
por encima del enunciado. Estas cuestiones pueden provenir de alguna
consulta con docentes. También pueden provenir de alguna especulación
o interpretación que el grupo hizo.
\noindent
\textbf{Tiempo de pedido:} asumimos que dos clientes no puede hacer un pedido de forma simultánea. Esta presunción permite simplificar el cumplimiento del objetivo relacionado a no dejar que se pueda cancelar un pedido a un cliente luego de que fue hecho, dado que establece un claro orden entre ellos. \\
\textbf{Tiempo de pedido a distancia:} asumimos que, ante un pedido a `distancia' (es decir, de un local a otro), un cliente pierde el interés en buscar dicho pedido luego de un \textit{time out} pautado de antemano. Esta presunción nos ayuda a poder cumplir, por un lado, el objetivo relacionado a permitir dicha compra `a distancia' y, por otro, a poder mejorar el manejo de stock al cancelar el pedido luego de pasado el \textit{time out}, y de esta forma recuperar el stock reservado para éste. Esto se logra sin violar los objetivos relacionados al funcionamiento de las ventas (no cancelar pedidos hechos), dado que se interpreta como que el cliente es el que cancela el pedido al no presentarse y no el local.


\section*{Vistas}
\addcontentsline{toc}{section}{Vistas}

Cosas que dijo la catedra:

está sería la parte principal del TP. Acá irían los
diagramas y los escenarios representativos de uso. No necesariamente
tiene que ser una gran sección, sino que pueden partirla por
funcionalidades y a su vez por tipo de vista o de la forma que crean
más pertinente. Esta sección NO DEBE SER una simple seguidilla de
figuras sueltas. Debe estar acompañada de tantas explicaciones como
sean necesarias para que se aprecie un hilo conductor.

\section*{Modelo de objetivos}
\addcontentsline{toc}{subsection}{Modelo de obejtivos}
\subsubsection*{Objetivos}
\addcontentsline{toc}{subsubsection}{Obejtivos}
\noindent
\textbf{Mantener[Menú estandarizado]:} para mantener un menú estandarizado simplemente tenemos que aseguranos que sea el mismo en todos los locales, sean cual sean las actualizaciones que se han realizado (de variedades, precios, etc). Para ello únicamente tenemos que asegurar que se mantenga la comunicación entre los locales. \\
\textbf{Lograr[Menú actualizable]:} para lograr un menú actualizable tenemos que asegurarnos de que se puedan realizar las actualizaciones pertinentes de éste. Para ello es necesario que haya un sistema de control de actualizaciones que resuelva el problema de ingresar dos actualizaciones en simultáneo, y por otro lado, se necesita que las actualizaciones pueda llegar a todos los locales (no es lo mismo que Mantener[Menú estandarizado]?.m En el modelo original estas dos ramas estaban separadas, talvez sea lo más conveniente para no entrar en redundancias). \\
\textbf{Lograr [Autorizar el ingreso de un menú nuevo desde una sucursal]:} pensamos varias formas de manejar esto: una de ellas consiste en que los locales pueden realizar actualizaciones en un momento determinado del día, bien puede ser esto a una hora prefijada de antemano o bien puede ser cuando cierra el local a la noche. Si el local no respeta estos horarios no le es permitido ingresar la actualización. Otro caso a considerar es cuando se cae la conexión del sistema, en cuyo caso tampoco se permiten las actualizaciones, dado que sería una actualización local y el menú debe ser el mismo en TODOS los lugares. \\
\textbf{Lograr [Recargar stock al detectarse el nivel de stock]:} para esto planteamos dos alternativas. La primera es recargar el stock cuando se agote completamente, que es más eficiente en el manejo de stock porque involucra que se carga sólo las veces necesarias (recordar que cada carga tene su costo, por ejemplo el pago de transporte al proveedor). La segunda es recargar el stock cuando se detecte bajo stock, esto no es eficiente en términos de manejo de stock, pero ayudaría enormemente a mantener un menú rico en disponibilidad de productos, lo cual conllevaría a perder menos clientes debido a que sus pedidos no pudieron ser satisfechos. \\
\textbf{Lograr [Cancelar pedido a distancia]:} para cumplir este objetivo planteamos dos alternativas: por un lado se permite al cliente cancelar el pedido a través de algún sistema de comunicación, mientras que por el otro, el local es el que cancela el pedido al excederse un cierto \textit{time out}. Esta última se apoya en una presunción del dominio que indica que un cliente pierde el interés en realizar la compra luego de pasado cierto tiempo, en cuyo caso se cancela el pedido de forma automática. El primer caso en cambio, da libertad al cliente de cancelar el pedido por su cuenta, llamando a la sucursal en cuestión y aclarando la situación. Pensamos que esta última opción de alguna forma mejora la satisfacción del cliente al no tener que obligarlo a ir hasta la sucursal para cancelar el pedido.


\section*{Discusi\'on}
\addcontentsline{toc}{section}{Discusi\'on}

Cosas que dijo la catedra:

debe contener un análisis general de las distintas
alternativas y su impacto en los objetivos blandos. NO DEBE SER una
lista de lo que ya se puede desprender del diagrama. Debe ser un
análisis más cualitativo y "a vuelo de pájaro" que permita entender
este asunto en pocas palabras. Esta sección también es ideal para que
vuelquen las cosas que puedan haberles quedado "flojas" o no cerradas
del todo: por ejemplo los potenciales conflictos entre objetivos, los
aspectos que podrían hacer que todo el sistema no funcione como
desean, etc.


\section*{Conclusiones}
\addcontentsline{toc}{section}{Conclusiones}

Cosas que dijo la catedra:

mencionar brevemente de qué formas les resultó más
sencillo encarar el TP. Por ejemplo en qué orden realizaron los
diagramas, qué aspectos presentaron las mayores dificultades, etc.




\end{document}
