\documentclass[a4paper,10pt]{article}
\usepackage{graphicx}
\usepackage{verbatim}
% \usepackage{lstlisting}
\usepackage{subfig}
\usepackage{float}
 \usepackage[spanish]{babel}   %ver bien como es
\usepackage[utf8]{inputenc}

\begin{document}

\tableofcontents

\newpage


\section*{Introducci\'on}
\addcontentsline{toc}{section}{Introducci\'on}

Cosas que dijo la catedra:

acá debería figurar todo lo necesario como para que un
lector no iniciado en el tema entienda el propósito general del
sistema que van a describir. Puede citar algunos fragmentos del
enunciado si lo consideran necesario, pero NO DEBE SER una copia del
enunciado.

\section*{Presunciones}
\addcontentsline{toc}{section}{Introducci\'on}

Cosas que dijo la catedra:

acá deberían listar aquellas cuestiones que asumieron
por encima del enunciado. Estas cuestiones pueden provenir de alguna
consulta con docentes. También pueden provenir de alguna especulación
o interpretación que el grupo hizo.


\section*{Vistas}
\addcontentsline{toc}{section}{Vistas}

Cosas que dijo la catedra:

está sería la parte principal del TP. Acá irían los
diagramas y los escenarios representativos de uso. No necesariamente
tiene que ser una gran sección, sino que pueden partirla por
funcionalidades y a su vez por tipo de vista o de la forma que crean
más pertinente. Esta sección NO DEBE SER una simple seguidilla de
figuras sueltas. Debe estar acompañada de tantas explicaciones como
sean necesarias para que se aprecie un hilo conductor.


\section*{Discusi\'on}
\addcontentsline{toc}{section}{Discusi\'on}

Cosas que dijo la catedra:

debe contener un análisis general de las distintas
alternativas y su impacto en los objetivos blandos. NO DEBE SER una
lista de lo que ya se puede desprender del diagrama. Debe ser un
análisis más cualitativo y "a vuelo de pájaro" que permita entender
este asunto en pocas palabras. Esta sección también es ideal para que
vuelquen las cosas que puedan haberles quedado "flojas" o no cerradas
del todo: por ejemplo los potenciales conflictos entre objetivos, los
aspectos que podrían hacer que todo el sistema no funcione como
desean, etc.


\section*{Conclusiones}
\addcontentsline{toc}{section}{Conclusiones}

Cosas que dijo la catedra:

mencionar brevemente de qué formas les resultó más
sencillo encarar el TP. Por ejemplo en qué orden realizaron los
diagramas, qué aspectos presentaron las mayores dificultades, etc.




\end{document}
